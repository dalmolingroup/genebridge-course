% Options for packages loaded elsewhere
\PassOptionsToPackage{unicode}{hyperref}
\PassOptionsToPackage{hyphens}{url}
\PassOptionsToPackage{dvipsnames,svgnames,x11names}{xcolor}
%
\documentclass[
  letterpaper,
  DIV=11,
  numbers=noendperiod]{scrreprt}

\usepackage{amsmath,amssymb}
\usepackage{iftex}
\ifPDFTeX
  \usepackage[T1]{fontenc}
  \usepackage[utf8]{inputenc}
  \usepackage{textcomp} % provide euro and other symbols
\else % if luatex or xetex
  \usepackage{unicode-math}
  \defaultfontfeatures{Scale=MatchLowercase}
  \defaultfontfeatures[\rmfamily]{Ligatures=TeX,Scale=1}
\fi
\usepackage{lmodern}
\ifPDFTeX\else  
    % xetex/luatex font selection
\fi
% Use upquote if available, for straight quotes in verbatim environments
\IfFileExists{upquote.sty}{\usepackage{upquote}}{}
\IfFileExists{microtype.sty}{% use microtype if available
  \usepackage[]{microtype}
  \UseMicrotypeSet[protrusion]{basicmath} % disable protrusion for tt fonts
}{}
\makeatletter
\@ifundefined{KOMAClassName}{% if non-KOMA class
  \IfFileExists{parskip.sty}{%
    \usepackage{parskip}
  }{% else
    \setlength{\parindent}{0pt}
    \setlength{\parskip}{6pt plus 2pt minus 1pt}}
}{% if KOMA class
  \KOMAoptions{parskip=half}}
\makeatother
\usepackage{xcolor}
\setlength{\emergencystretch}{3em} % prevent overfull lines
\setcounter{secnumdepth}{5}
% Make \paragraph and \subparagraph free-standing
\ifx\paragraph\undefined\else
  \let\oldparagraph\paragraph
  \renewcommand{\paragraph}[1]{\oldparagraph{#1}\mbox{}}
\fi
\ifx\subparagraph\undefined\else
  \let\oldsubparagraph\subparagraph
  \renewcommand{\subparagraph}[1]{\oldsubparagraph{#1}\mbox{}}
\fi

\usepackage{color}
\usepackage{fancyvrb}
\newcommand{\VerbBar}{|}
\newcommand{\VERB}{\Verb[commandchars=\\\{\}]}
\DefineVerbatimEnvironment{Highlighting}{Verbatim}{commandchars=\\\{\}}
% Add ',fontsize=\small' for more characters per line
\usepackage{framed}
\definecolor{shadecolor}{RGB}{241,243,245}
\newenvironment{Shaded}{\begin{snugshade}}{\end{snugshade}}
\newcommand{\AlertTok}[1]{\textcolor[rgb]{0.68,0.00,0.00}{#1}}
\newcommand{\AnnotationTok}[1]{\textcolor[rgb]{0.37,0.37,0.37}{#1}}
\newcommand{\AttributeTok}[1]{\textcolor[rgb]{0.40,0.45,0.13}{#1}}
\newcommand{\BaseNTok}[1]{\textcolor[rgb]{0.68,0.00,0.00}{#1}}
\newcommand{\BuiltInTok}[1]{\textcolor[rgb]{0.00,0.23,0.31}{#1}}
\newcommand{\CharTok}[1]{\textcolor[rgb]{0.13,0.47,0.30}{#1}}
\newcommand{\CommentTok}[1]{\textcolor[rgb]{0.37,0.37,0.37}{#1}}
\newcommand{\CommentVarTok}[1]{\textcolor[rgb]{0.37,0.37,0.37}{\textit{#1}}}
\newcommand{\ConstantTok}[1]{\textcolor[rgb]{0.56,0.35,0.01}{#1}}
\newcommand{\ControlFlowTok}[1]{\textcolor[rgb]{0.00,0.23,0.31}{#1}}
\newcommand{\DataTypeTok}[1]{\textcolor[rgb]{0.68,0.00,0.00}{#1}}
\newcommand{\DecValTok}[1]{\textcolor[rgb]{0.68,0.00,0.00}{#1}}
\newcommand{\DocumentationTok}[1]{\textcolor[rgb]{0.37,0.37,0.37}{\textit{#1}}}
\newcommand{\ErrorTok}[1]{\textcolor[rgb]{0.68,0.00,0.00}{#1}}
\newcommand{\ExtensionTok}[1]{\textcolor[rgb]{0.00,0.23,0.31}{#1}}
\newcommand{\FloatTok}[1]{\textcolor[rgb]{0.68,0.00,0.00}{#1}}
\newcommand{\FunctionTok}[1]{\textcolor[rgb]{0.28,0.35,0.67}{#1}}
\newcommand{\ImportTok}[1]{\textcolor[rgb]{0.00,0.46,0.62}{#1}}
\newcommand{\InformationTok}[1]{\textcolor[rgb]{0.37,0.37,0.37}{#1}}
\newcommand{\KeywordTok}[1]{\textcolor[rgb]{0.00,0.23,0.31}{#1}}
\newcommand{\NormalTok}[1]{\textcolor[rgb]{0.00,0.23,0.31}{#1}}
\newcommand{\OperatorTok}[1]{\textcolor[rgb]{0.37,0.37,0.37}{#1}}
\newcommand{\OtherTok}[1]{\textcolor[rgb]{0.00,0.23,0.31}{#1}}
\newcommand{\PreprocessorTok}[1]{\textcolor[rgb]{0.68,0.00,0.00}{#1}}
\newcommand{\RegionMarkerTok}[1]{\textcolor[rgb]{0.00,0.23,0.31}{#1}}
\newcommand{\SpecialCharTok}[1]{\textcolor[rgb]{0.37,0.37,0.37}{#1}}
\newcommand{\SpecialStringTok}[1]{\textcolor[rgb]{0.13,0.47,0.30}{#1}}
\newcommand{\StringTok}[1]{\textcolor[rgb]{0.13,0.47,0.30}{#1}}
\newcommand{\VariableTok}[1]{\textcolor[rgb]{0.07,0.07,0.07}{#1}}
\newcommand{\VerbatimStringTok}[1]{\textcolor[rgb]{0.13,0.47,0.30}{#1}}
\newcommand{\WarningTok}[1]{\textcolor[rgb]{0.37,0.37,0.37}{\textit{#1}}}

\providecommand{\tightlist}{%
  \setlength{\itemsep}{0pt}\setlength{\parskip}{0pt}}\usepackage{longtable,booktabs,array}
\usepackage{calc} % for calculating minipage widths
% Correct order of tables after \paragraph or \subparagraph
\usepackage{etoolbox}
\makeatletter
\patchcmd\longtable{\par}{\if@noskipsec\mbox{}\fi\par}{}{}
\makeatother
% Allow footnotes in longtable head/foot
\IfFileExists{footnotehyper.sty}{\usepackage{footnotehyper}}{\usepackage{footnote}}
\makesavenoteenv{longtable}
\usepackage{graphicx}
\makeatletter
\def\maxwidth{\ifdim\Gin@nat@width>\linewidth\linewidth\else\Gin@nat@width\fi}
\def\maxheight{\ifdim\Gin@nat@height>\textheight\textheight\else\Gin@nat@height\fi}
\makeatother
% Scale images if necessary, so that they will not overflow the page
% margins by default, and it is still possible to overwrite the defaults
% using explicit options in \includegraphics[width, height, ...]{}
\setkeys{Gin}{width=\maxwidth,height=\maxheight,keepaspectratio}
% Set default figure placement to htbp
\makeatletter
\def\fps@figure{htbp}
\makeatother

\usepackage{fvextra}
\DefineVerbatimEnvironment{Highlighting}{Verbatim}{breaklines,commandchars=\\\{\}}
\KOMAoption{captions}{tableheading}
\makeatletter
\@ifpackageloaded{bookmark}{}{\usepackage{bookmark}}
\makeatother
\makeatletter
\@ifpackageloaded{caption}{}{\usepackage{caption}}
\AtBeginDocument{%
\ifdefined\contentsname
  \renewcommand*\contentsname{Table of contents}
\else
  \newcommand\contentsname{Table of contents}
\fi
\ifdefined\listfigurename
  \renewcommand*\listfigurename{List of Figures}
\else
  \newcommand\listfigurename{List of Figures}
\fi
\ifdefined\listtablename
  \renewcommand*\listtablename{List of Tables}
\else
  \newcommand\listtablename{List of Tables}
\fi
\ifdefined\figurename
  \renewcommand*\figurename{Figure}
\else
  \newcommand\figurename{Figure}
\fi
\ifdefined\tablename
  \renewcommand*\tablename{Table}
\else
  \newcommand\tablename{Table}
\fi
}
\@ifpackageloaded{float}{}{\usepackage{float}}
\floatstyle{ruled}
\@ifundefined{c@chapter}{\newfloat{codelisting}{h}{lop}}{\newfloat{codelisting}{h}{lop}[chapter]}
\floatname{codelisting}{Listing}
\newcommand*\listoflistings{\listof{codelisting}{List of Listings}}
\makeatother
\makeatletter
\makeatother
\makeatletter
\@ifpackageloaded{caption}{}{\usepackage{caption}}
\@ifpackageloaded{subcaption}{}{\usepackage{subcaption}}
\makeatother
\ifLuaTeX
  \usepackage{selnolig}  % disable illegal ligatures
\fi
\usepackage{bookmark}

\IfFileExists{xurl.sty}{\usepackage{xurl}}{} % add URL line breaks if available
\urlstyle{same} % disable monospaced font for URLs
\hypersetup{
  pdftitle={GeneBridge: Inferring the Evolutionary Rooting of Orthologous Genes},
  pdfauthor={Dalmolin Systems Biology Group},
  colorlinks=true,
  linkcolor={blue},
  filecolor={Maroon},
  citecolor={Blue},
  urlcolor={Blue},
  pdfcreator={LaTeX via pandoc}}

\title{GeneBridge: Inferring the Evolutionary Rooting of Orthologous
Genes}
\author{Dalmolin Systems Biology Group}
\date{}

\begin{document}
\maketitle

\RecustomVerbatimEnvironment{verbatim}{Verbatim}{
  showspaces = false,
  showtabs = false,
  breaksymbolleft={},
  breaklines
  % Note: setting commandchars=\\\{\} here will cause an error 
}

\renewcommand*\contentsname{Table of contents}
{
\hypersetup{linkcolor=}
\setcounter{tocdepth}{2}
\tableofcontents
}
\bookmarksetup{startatroot}

\chapter{GeneBridge}\label{genebridge}

\bookmarksetup{startatroot}

\chapter{GeneBridge: Inferring the Evolutionary Rooting of Orthologous
Genes}\label{genebridge-inferring-the-evolutionary-rooting-of-orthologous-genes}

\section{Lesson overview}\label{lesson-overview}

🔒 \textbf{License:}
\href{https://creativecommons.org/licenses/by-sa/4.0/deed.en}{Creative
Commons Attribution Share Alike 4.0 International License}

👥 \textbf{Target Audience:} Researchers, graduate students, and
biologists interested in evolutionary analysis.

📶 \textbf{Level:} Intermediate

⬅️ \textbf{Prerequisites} To be able to follow this course, learners
should have knowledge in:

\begin{enumerate}
\def\labelenumi{\arabic{enumi}.}
\tightlist
\item
  Basic knowledge of the R programming language (loading libraries,
  handling data frames).
\item
  Familiarity with fundamental concepts in evolutionary biology (e.g.,
  genes, orthologs, phylogenetic trees).
\item
  Being comfortable working with RStudio or a similar R environment.
\end{enumerate}

📖 \textbf{Description} This repository contains the materials for the
course ``GeneBridge: Inferring the Evolutionary Rooting of Orthologous
Genes,'' organized by from Dalmolin's Systems Biology Group, based at
the Bioinformatics Multidisciplinary Environment (BioME) at UFRN. This
workshop is divided into three parts: (1) Performing the Rooting
Analysis, (2) Plotting Rooted Genes, and (3) Plotting Abundances.

➡️ \textbf{Learning Outcomes:} By the end of the course, learners will
be able to:

\begin{enumerate}
\def\labelenumi{\arabic{enumi}.}
\tightlist
\item
  \textbf{Understand} the theoretical basis of evolutionary rooting and
  the Bridge algorithm. {[}Understanding{]}
\item
  \textbf{Apply} the GeneBridge package to infer the evolutionary root
  for a set of orthologous genes. {[}Applying{]}
\item
  \textbf{Analyze} and \textbf{interpret} the output tables and rooting
  results. {[}Analysing{]}
\item
  \textbf{Create} and \textbf{customize} plots to visualize rooted genes
  on a phylogenetic tree. {[}Creating{]}
\item
  \textbf{Generate} and \textbf{evaluate} plots of gene abundances
  across different clades. {[}Evaluating{]}
\end{enumerate}

⌛ \textbf{Time estimation}: 180 minutes

⚙️ \textbf{Requirements:} Participants must have a laptop with a recent
version of R and RStudio installed. Specific R package dependencies
(e.g., GeneBridge) are detailed in the repository's setup instructions.

🙏 \textbf{Acknowledgements}:

\begin{itemize}
\tightlist
\item
  Bioinformatics Multidisciplinary Environment (BioME - IMD/UFRN)
\item
  Postgraduate Program in Bioinformatics (PPg-Bioinfo - UFRN)
\end{itemize}

\bookmarksetup{startatroot}

\chapter{Gene Rooting}\label{gene-rooting}

\section{Importing Packages}\label{importing-packages}

\begin{Shaded}
\begin{Highlighting}[]
\FunctionTok{library}\NormalTok{(GeneBridge)}
\FunctionTok{library}\NormalTok{(geneplast.data)}
\FunctionTok{library}\NormalTok{(readr)}
\FunctionTok{library}\NormalTok{(dplyr)}
\FunctionTok{library}\NormalTok{(purrr)}
\FunctionTok{library}\NormalTok{(biomaRt)}
\FunctionTok{library}\NormalTok{(magrittr)}
\FunctionTok{library}\NormalTok{(KEGGREST)}
\FunctionTok{library}\NormalTok{(ape)}
\FunctionTok{library}\NormalTok{(tidyverse)}
\FunctionTok{library}\NormalTok{(data.table)}
\FunctionTok{library}\NormalTok{(stringi)}
\FunctionTok{library}\NormalTok{(AnnotationHub)}
\FunctionTok{library}\NormalTok{(sourcetools)}
\FunctionTok{library}\NormalTok{(here)}
\end{Highlighting}
\end{Shaded}

\section{Defining functions for later
use}\label{defining-functions-for-later-use}

The first one searches for the respective protein ID for each gene in
the input list. The second returns the interactions between these
proteins, and the last one filters the interactions by a combined
confidence score greater than 0.4.

\begin{Shaded}
\begin{Highlighting}[]
\CommentTok{\# get IDs from STRING DB}
\NormalTok{get\_string\_ids }\OtherTok{\textless{}{-}} \ControlFlowTok{function}\NormalTok{(genes\_hgnc, }\AttributeTok{species\_id =} \StringTok{"9606"}\NormalTok{) \{}

\NormalTok{    req }\OtherTok{\textless{}{-}}\NormalTok{ RCurl}\SpecialCharTok{::}\FunctionTok{postForm}\NormalTok{(}
    \StringTok{"https://string{-}db.org/api/tsv/get\_string\_ids"}\NormalTok{,}
    \AttributeTok{identifiers =} \FunctionTok{paste}\NormalTok{(genes\_hgnc, }\AttributeTok{collapse =} \StringTok{"\%0D"}\NormalTok{),}
    \AttributeTok{echo\_query =} \StringTok{"1"}\NormalTok{,}
    \AttributeTok{species =}\NormalTok{ species\_id,}
    \AttributeTok{.opts =} \FunctionTok{list}\NormalTok{(}\AttributeTok{ssl.verifypeer =} \ConstantTok{FALSE}\NormalTok{)}
\NormalTok{  )}
  
\NormalTok{  map\_ids }\OtherTok{\textless{}{-}} \FunctionTok{read.table}\NormalTok{(}\AttributeTok{text =}\NormalTok{ req, }\AttributeTok{sep =} \StringTok{" "}\NormalTok{, }\AttributeTok{header =} \ConstantTok{TRUE}\NormalTok{, }\AttributeTok{quote =} \StringTok{""}\NormalTok{) }\SpecialCharTok{\%\textgreater{}\%}
\NormalTok{    dplyr}\SpecialCharTok{::}\FunctionTok{select}\NormalTok{(}\SpecialCharTok{{-}}\NormalTok{queryIndex) }\SpecialCharTok{\%\textgreater{}\%}
    \FunctionTok{unique}\NormalTok{()}
  
\NormalTok{  map\_ids}\SpecialCharTok{$}\NormalTok{stringId }\OtherTok{\textless{}{-}} \FunctionTok{substring}\NormalTok{(map\_ids}\SpecialCharTok{$}\NormalTok{stringId, }\DecValTok{6}\NormalTok{, }\DecValTok{1000}\NormalTok{)}
  
  \FunctionTok{return}\NormalTok{(map\_ids)}
\NormalTok{\}}

 \CommentTok{\# Get STRING interactions}
\NormalTok{ get\_network\_interaction }\OtherTok{\textless{}{-}} \ControlFlowTok{function}\NormalTok{(map\_ids, protein\_id, }\AttributeTok{species\_id =} \StringTok{"9606"}\NormalTok{) \{}
 
\NormalTok{   identifiers }\OtherTok{\textless{}{-}}\NormalTok{ map\_ids }\SpecialCharTok{\%\textgreater{}\%} \FunctionTok{pull}\NormalTok{(protein\_id) }\SpecialCharTok{\%\textgreater{}\%}\NormalTok{ na.omit }\SpecialCharTok{\%\textgreater{}\%} \FunctionTok{paste0}\NormalTok{(}\AttributeTok{collapse=}\StringTok{"\%0d"}\NormalTok{)}
 
\NormalTok{   req2 }\OtherTok{\textless{}{-}}\NormalTok{ RCurl}\SpecialCharTok{::}\FunctionTok{postForm}\NormalTok{(}
     \StringTok{"https://string{-}db.org/api/tsv/network"}\NormalTok{,}
     \AttributeTok{identifiers =}\NormalTok{ identifiers,}
     \AttributeTok{required\_core =} \StringTok{"0"}\NormalTok{,}
     \AttributeTok{species =}\NormalTok{ species\_id,}
     \AttributeTok{.opts =} \FunctionTok{list}\NormalTok{(}\AttributeTok{ssl.verifypeer =} \ConstantTok{FALSE}\NormalTok{)}
\NormalTok{   )}
   
\NormalTok{   int\_network }\OtherTok{\textless{}{-}} \FunctionTok{read.table}\NormalTok{(}\AttributeTok{text =}\NormalTok{ req2, }\AttributeTok{sep =} \StringTok{"   "}\NormalTok{, }\AttributeTok{header =} \ConstantTok{TRUE}\NormalTok{)}
   
\NormalTok{   int\_network }\OtherTok{\textless{}{-}} \FunctionTok{unique}\NormalTok{(int\_network)}
   
   \FunctionTok{return}\NormalTok{(int\_network)}
\NormalTok{ \}}

 \DocumentationTok{\#\# Recomputing scores}
\NormalTok{combine\_scores }\OtherTok{\textless{}{-}} \ControlFlowTok{function}\NormalTok{(dat, }\AttributeTok{evidences =} \StringTok{"all"}\NormalTok{, }\AttributeTok{confLevel =} \FloatTok{0.4}\NormalTok{) \{}
  \ControlFlowTok{if}\NormalTok{(evidences[}\DecValTok{1}\NormalTok{] }\SpecialCharTok{==} \StringTok{"all"}\NormalTok{)\{}
\NormalTok{    edat}\OtherTok{\textless{}{-}}\NormalTok{dat[,}\SpecialCharTok{{-}}\FunctionTok{c}\NormalTok{(}\DecValTok{1}\NormalTok{,}\DecValTok{2}\NormalTok{,}\FunctionTok{ncol}\NormalTok{(dat))]}
\NormalTok{  \} }\ControlFlowTok{else}\NormalTok{ \{}
    \ControlFlowTok{if}\NormalTok{(}\SpecialCharTok{!}\FunctionTok{all}\NormalTok{(evidences}\SpecialCharTok{\%in\%}\FunctionTok{colnames}\NormalTok{(dat)))\{}
      \FunctionTok{stop}\NormalTok{(}\StringTok{"NOTE: one or more \textquotesingle{}evidences\textquotesingle{} not listed in \textquotesingle{}dat\textquotesingle{} colnames!"}\NormalTok{)}
\NormalTok{    \}}
\NormalTok{    edat}\OtherTok{\textless{}{-}}\NormalTok{dat[,evidences]}
\NormalTok{  \}}
  \ControlFlowTok{if}\NormalTok{ (}\FunctionTok{any}\NormalTok{(edat }\SpecialCharTok{\textgreater{}} \DecValTok{1}\NormalTok{)) \{}
\NormalTok{    edat }\OtherTok{\textless{}{-}}\NormalTok{ edat}\SpecialCharTok{/}\DecValTok{1000}
\NormalTok{  \}}
\NormalTok{  edat}\OtherTok{\textless{}{-}}\DecValTok{1}\SpecialCharTok{{-}}\NormalTok{edat}
\NormalTok{  sc}\OtherTok{\textless{}{-}} \FunctionTok{apply}\NormalTok{(}\AttributeTok{X =}\NormalTok{ edat, }\AttributeTok{MARGIN =} \DecValTok{1}\NormalTok{, }\AttributeTok{FUN =} \ControlFlowTok{function}\NormalTok{(x) }\DecValTok{1}\SpecialCharTok{{-}}\FunctionTok{prod}\NormalTok{(x))}
\NormalTok{  dat }\OtherTok{\textless{}{-}} \FunctionTok{cbind}\NormalTok{(dat[,}\FunctionTok{c}\NormalTok{(}\DecValTok{1}\NormalTok{,}\DecValTok{2}\NormalTok{)],}\AttributeTok{combined\_score =}\NormalTok{ sc)}
\NormalTok{  idx }\OtherTok{\textless{}{-}}\NormalTok{ dat}\SpecialCharTok{$}\NormalTok{combined\_score }\SpecialCharTok{\textgreater{}=}\NormalTok{ confLevel}
\NormalTok{  dat }\OtherTok{\textless{}{-}}\NormalTok{dat[idx,]}
  \FunctionTok{return}\NormalTok{(dat)}
\NormalTok{\}}
\end{Highlighting}
\end{Shaded}

\section{Loading gene list and orthology
data}\label{loading-gene-list-and-orthology-data}

We load the orthology data through \textbf{AnnotationHub}, this R
package provides a central place where genomic files (VCF, bed, wig) and
other resources from standard locations (e.g., UCSC, Ensembl) can be
accessed. This way, we have access to the input files for the GeneBridge
algorithm.

\begin{Shaded}
\begin{Highlighting}[]
\CommentTok{\# Load the Gene Set Table}
\NormalTok{sensorial\_genes }\OtherTok{\textless{}{-}} \FunctionTok{read.csv}\NormalTok{(}\FunctionTok{here}\NormalTok{(}\StringTok{"data/sensorial\_genes.csv"}\NormalTok{))}

\CommentTok{\# Query Phylotree and OG data}
\NormalTok{ah }\OtherTok{\textless{}{-}} \FunctionTok{AnnotationHub}\NormalTok{()}
\NormalTok{meta }\OtherTok{\textless{}{-}} \FunctionTok{query}\NormalTok{(ah, }\StringTok{"geneplast"}\NormalTok{)}
\FunctionTok{load}\NormalTok{(meta[[}\StringTok{"AH83116"}\NormalTok{]])}

\FunctionTok{head}\NormalTok{(sensorial\_genes)}
\FunctionTok{head}\NormalTok{(cogdata)}
\end{Highlighting}
\end{Shaded}

\section{1. Pre-processing}\label{pre-processing}

\subsection{1.1. Mapping}\label{mapping}

For the next analyses, we need to cross-reference information between
our genes of interest (Gene IDs from the
\textbf{\texttt{sensorial\_genes}} table) and Protein IDs (from the
\textbf{\texttt{cogdata}} table). The STRINGdb API is used to map Gene
IDs to Protein IDs, allowing the filtering of genes of interest in the
cogdata table. The final goal is to obtain a filtered set of sensory
genes with their respective pathways and COG IDs.

\begin{Shaded}
\begin{Highlighting}[]
\NormalTok{map\_ids }\OtherTok{\textless{}{-}} \FunctionTok{get\_string\_ids}\NormalTok{(sensorial\_genes}\SpecialCharTok{$}\NormalTok{gene\_symbol)}

\CommentTok{\# Subsetting cogs of interest {-} Sensorial Genes}
\NormalTok{gene\_cogs }\OtherTok{\textless{}{-}}\NormalTok{ cogdata }\SpecialCharTok{\%\textgreater{}\%}
  \FunctionTok{filter}\NormalTok{(ssp\_id }\SpecialCharTok{\%in\%}\NormalTok{ map\_ids}\SpecialCharTok{$}\NormalTok{ncbiTaxonId) }\SpecialCharTok{\%\textgreater{}\%}
  \FunctionTok{filter}\NormalTok{(protein\_id }\SpecialCharTok{\%in\%}\NormalTok{ map\_ids[[}\StringTok{"stringId"}\NormalTok{]]) }\SpecialCharTok{\%\textgreater{}\%}
  \FunctionTok{group\_by}\NormalTok{(protein\_id) }\SpecialCharTok{\%\textgreater{}\%}
  \FunctionTok{summarise}\NormalTok{(}\AttributeTok{n =} \FunctionTok{n}\NormalTok{(), }\AttributeTok{cog\_id =} \FunctionTok{paste}\NormalTok{(cog\_id, }\AttributeTok{collapse =} \StringTok{" / "}\NormalTok{))}

\FunctionTok{head}\NormalTok{(map\_ids)}

\CommentTok{\#map\_ids |\textgreater{} }
\CommentTok{\#  vroom::vroom\_write(file = here("data/map\_ids.csv"), delim = ",")}
\end{Highlighting}
\end{Shaded}

\subsection{1.2. Resolving duplicate
COGs}\label{resolving-duplicate-cogs}

Due to evolutionary events such as gene duplication, some genes may be
associated with more than one Cluster of Orthologous Groups (COG). To
ensure the functionality of the algorithm, it is necessary to resolve
these cases, prioritizing COGs according to the following criteria:

\begin{enumerate}
\def\labelenumi{\arabic{enumi}.}
\tightlist
\item
  Priority by COG type:
\end{enumerate}

\begin{itemize}
\tightlist
\item
  KOGs have higher priority.
\item
  COGs have higher priority than NOGs.
\end{itemize}

\begin{enumerate}
\def\labelenumi{\arabic{enumi}.}
\setcounter{enumi}{1}
\tightlist
\item
  Cases with COGs starting with the same letter:
\end{enumerate}

\begin{itemize}
\tightlist
\item
  Are resolved manually, based on the annotated function of the COG and
  the scientific question of the study.
\end{itemize}

The code below implements this resolution and integrates the corrections
into the main table.

\begin{Shaded}
\begin{Highlighting}[]
\NormalTok{gene\_cogs }\SpecialCharTok{\%\textgreater{}\%} \FunctionTok{filter}\NormalTok{(n }\SpecialCharTok{\textgreater{}} \DecValTok{1}\NormalTok{)}

\CommentTok{\# Resolving main proteins}
\NormalTok{gene\_cogs\_resolved }\OtherTok{\textless{}{-}} \FunctionTok{tribble}\NormalTok{(}
  \SpecialCharTok{\textasciitilde{}}\NormalTok{protein\_id, }\SpecialCharTok{\textasciitilde{}}\NormalTok{cog\_id,}
\StringTok{"ENSP00000332500"}\NormalTok{, }\StringTok{"NOG274749"}\NormalTok{, }\CommentTok{\#NOG274749 / NOG274749}
\StringTok{"ENSP00000409316"}\NormalTok{, }\StringTok{"NOG282909"}\NormalTok{, }\CommentTok{\#NOG282909 / NOG282909 / NOG282909}
\StringTok{"ENSP00000480090"}\NormalTok{, }\StringTok{"KOG3599"}    \CommentTok{\#KOG3599 / KOG3272}
\NormalTok{)}

\CommentTok{\# Removing unresolved cases and adding manual assignments}
\NormalTok{gene\_cogs }\SpecialCharTok{\%\textless{}\textgreater{}\%}
  \FunctionTok{filter}\NormalTok{(n }\SpecialCharTok{==} \DecValTok{1}\NormalTok{) }\SpecialCharTok{\%\textgreater{}\%}
\NormalTok{ dplyr}\SpecialCharTok{::} \FunctionTok{select}\NormalTok{(}\SpecialCharTok{{-}}\NormalTok{n) }\SpecialCharTok{\%\textgreater{}\%}
  \FunctionTok{bind\_rows}\NormalTok{(gene\_cogs\_resolved)}

\CommentTok{\#gene\_cogs |\textgreater{} }
\CommentTok{\#  vroom::vroom\_write(file = here("data/gene\_cogs.csv"), delim = ",")}
\end{Highlighting}
\end{Shaded}

\section{3. Processing}\label{processing}

The objective of this step is to perform the rooting of the genes of
interest using the \textbf{GeneBridge} package. For this, we use the
\texttt{newBridge}, \texttt{runBridge}, and \texttt{runPermutation}
functions, which produce statistical results associated with the
selected COGs in a phylogenetic tree.

\subsection{3.1. Necessary Inputs}\label{necessary-inputs}

\begin{enumerate}
\def\labelenumi{\arabic{enumi}.}
\tightlist
\item
  \textbf{\texttt{ogdata}}:

  \begin{itemize}
  \tightlist
  \item
    Dataset containing three main columns:

    \begin{itemize}
    \tightlist
    \item
      \texttt{Protein\ ID}: Protein identifiers.
    \item
      \texttt{COG\ ID}: Clusters of interest.
    \item
      \texttt{Specie\ ID}: Species identifiers.
    \end{itemize}
  \item
    In the example, the \texttt{cogdata} object is being used.
  \end{itemize}
\item
  \textbf{\texttt{phyloTree}}:

  \begin{itemize}
  \tightlist
  \item
    Phylogenetic tree containing 476 eukaryotes, representing the
    evolutionary structure among the analyzed species.
  \end{itemize}
\item
  \textbf{\texttt{ogids}}:

  \begin{itemize}
  \tightlist
  \item
    List of \textbf{COGs of interest}. This set is derived from the
    \texttt{gene\_cogs} table and includes the COGs associated with the
    proteins after the previous processing.
  \end{itemize}
\item
  \textbf{\texttt{refsp}}:

  \begin{itemize}
  \tightlist
  \item
    Reference species for rooting. In the example, we use \texttt{9606}
    (human).
  \end{itemize}
\end{enumerate}

The \emph{getBridge} function extracts the results generated by
GeneBridge in table format. The \emph{res} table contains the
statistical results of the rooting.

\begin{Shaded}
\begin{Highlighting}[]
\DocumentationTok{\#\# Run GeneBridge}
\NormalTok{cogs\_of\_interest }\OtherTok{\textless{}{-}}\NormalTok{ gene\_cogs }\SpecialCharTok{\%\textgreater{}\%} \FunctionTok{pull}\NormalTok{(cog\_id) }\SpecialCharTok{\%\textgreater{}\%}\NormalTok{ unique}

\NormalTok{ogr }\OtherTok{\textless{}{-}} \FunctionTok{newBridge}\NormalTok{(}\AttributeTok{ogdata=}\NormalTok{cogdata, }\AttributeTok{phyloTree=}\NormalTok{phyloTree, }\AttributeTok{ogids =}\NormalTok{ cogs\_of\_interest, }\AttributeTok{refsp=}\StringTok{"9606"}\NormalTok{)}

\NormalTok{ogr }\OtherTok{\textless{}{-}} \FunctionTok{runBridge}\NormalTok{(ogr, }\AttributeTok{penalty =} \DecValTok{2}\NormalTok{, }\AttributeTok{threshold =} \FloatTok{0.5}\NormalTok{, }\AttributeTok{verbose =} \ConstantTok{TRUE}\NormalTok{)}

\NormalTok{ogr }\OtherTok{\textless{}{-}} \FunctionTok{runPermutation}\NormalTok{(ogr, }\AttributeTok{nPermutations=}\DecValTok{1000}\NormalTok{, }\AttributeTok{verbose=}\ConstantTok{FALSE}\NormalTok{)}

\NormalTok{res }\OtherTok{\textless{}{-}} \FunctionTok{getBridge}\NormalTok{(ogr, }\AttributeTok{what=}\StringTok{"results"}\NormalTok{)}

\FunctionTok{saveRDS}\NormalTok{(ogr, }\AttributeTok{file =} \FunctionTok{here}\NormalTok{(}\StringTok{"data/ogr.RData"}\NormalTok{))}
\end{Highlighting}
\end{Shaded}

\section{4. Post-processing}\label{post-processing}

After performing the rooting with \textbf{GeneBridge}, it is necessary
to adjust the data to improve the visualization and interpretation of
the results. In this step, we add the names of the clades to the
identified roots, using an external table that relates the root
identifiers to the clade names.

\begin{Shaded}
\begin{Highlighting}[]
\CommentTok{\# naming the rooted clades}
\NormalTok{CLADE\_NAMES }\OtherTok{\textless{}{-}} \StringTok{"https://raw.githubusercontent.com/dalmolingroup/neurotransmissionevolution/ctenophora\_before\_porifera/analysis/geneplast\_clade\_names.tsv"}

\NormalTok{lca\_names }\OtherTok{\textless{}{-}}\NormalTok{ vroom}\SpecialCharTok{::}\FunctionTok{vroom}\NormalTok{(CLADE\_NAMES)}

\NormalTok{groot\_df }\OtherTok{\textless{}{-}}\NormalTok{ res }\SpecialCharTok{\%\textgreater{}\%}
\NormalTok{  tibble}\SpecialCharTok{::}\FunctionTok{rownames\_to\_column}\NormalTok{(}\StringTok{"cog\_id"}\NormalTok{) }\SpecialCharTok{\%\textgreater{}\%}
\NormalTok{  dplyr}\SpecialCharTok{::}\FunctionTok{select}\NormalTok{(cog\_id, }\AttributeTok{root =}\NormalTok{ Root) }\SpecialCharTok{\%\textgreater{}\%}
  \FunctionTok{left\_join}\NormalTok{(lca\_names) }\SpecialCharTok{\%\textgreater{}\%}
  \FunctionTok{inner\_join}\NormalTok{(gene\_cogs)}

\FunctionTok{head}\NormalTok{(groot\_df)}

\CommentTok{\#groot\_df |\textgreater{} }
\CommentTok{\#  vroom::vroom\_write(file = here("data/groot\_df.csv"), delim = ",")}
\end{Highlighting}
\end{Shaded}

\subsection{4.1. Protein-Protein Interaction
Network}\label{protein-protein-interaction-network}

The construction of a protein-protein interaction (PPI) network is an
essential step to identify the functional relationships between
proteins. In this process, we use the \textbf{STRINGdb} API, a database
that catalogs interactions between proteins based on various sources,
including experimental assays, co-expression, and evidence extracted
from scientific publications.

The STRINGdb API offers methods to: - Obtain protein interactions for a
list of proteins. - Select specific sources of evidence. - Calculate and
combine scores based on the selected evidence.

More information about the API can be found in the
\href{https://string-db.org/help/api/}{STRING API documentation}.

\begin{Shaded}
\begin{Highlighting}[]
\CommentTok{\# Get proteins interaction}
\NormalTok{string\_edgelist }\OtherTok{\textless{}{-}} \FunctionTok{get\_network\_interaction}\NormalTok{(groot\_df)}

\CommentTok{\# Recomputing scores}
\NormalTok{string\_edgelist }\OtherTok{\textless{}{-}} \FunctionTok{combine\_scores}\NormalTok{(string\_edgelist,}
                                  \AttributeTok{evidences =} \FunctionTok{c}\NormalTok{(}\StringTok{"ascore"}\NormalTok{, }\StringTok{"escore"}\NormalTok{, }\StringTok{"dscore"}\NormalTok{),}
                                  \AttributeTok{confLevel =} \FloatTok{0.7}\NormalTok{)}

\FunctionTok{colnames}\NormalTok{(string\_edgelist) }\OtherTok{\textless{}{-}} \FunctionTok{c}\NormalTok{(}\StringTok{"stringId\_A"}\NormalTok{, }\StringTok{"stringId\_B"}\NormalTok{, }\StringTok{"combined\_score"}\NormalTok{)}

\CommentTok{\# Remove the species id}
\NormalTok{string\_edgelist}\SpecialCharTok{$}\NormalTok{stringId\_A }\OtherTok{\textless{}{-}} \FunctionTok{substring}\NormalTok{(string\_edgelist}\SpecialCharTok{$}\NormalTok{stringId\_A, }\DecValTok{6}\NormalTok{, }\DecValTok{1000}\NormalTok{)}
\NormalTok{string\_edgelist}\SpecialCharTok{$}\NormalTok{stringId\_B }\OtherTok{\textless{}{-}} \FunctionTok{substring}\NormalTok{(string\_edgelist}\SpecialCharTok{$}\NormalTok{stringId\_B, }\DecValTok{6}\NormalTok{, }\DecValTok{1000}\NormalTok{)}

\CommentTok{\# How many edgelist proteins are absent in gene\_ids? (should return 0)}
\FunctionTok{setdiff}\NormalTok{(}
\NormalTok{  string\_edgelist }\SpecialCharTok{\%$\%} \FunctionTok{c}\NormalTok{(stringId\_A, stringId\_B),}
\NormalTok{  map\_ids }\SpecialCharTok{\%\textgreater{}\%} \FunctionTok{pull}\NormalTok{(stringId)}
\NormalTok{)}

\FunctionTok{head}\NormalTok{(string\_edgelist)}
\end{Highlighting}
\end{Shaded}

For the construction of the graph, in addition to the interactions
between the proteins, it is necessary that each node is annotated with
additional information that will be used in the analysis, such as: -
Protein name. - Clade where it is rooted. - Metabolic pathway in which
it participates.

\begin{Shaded}
\begin{Highlighting}[]
\DocumentationTok{\#\# Create anotation table}
\NormalTok{nodelist }\OtherTok{\textless{}{-}} \FunctionTok{data.frame}\NormalTok{(}\AttributeTok{node =} \FunctionTok{unique}\NormalTok{(}\FunctionTok{c}\NormalTok{(string\_edgelist}\SpecialCharTok{$}\NormalTok{stringId\_A, string\_edgelist}\SpecialCharTok{$}\NormalTok{stringId\_B)))}

\NormalTok{merged\_paths }\OtherTok{\textless{}{-}} \FunctionTok{merge}\NormalTok{(nodelist, groot\_df, }\AttributeTok{by.x =} \StringTok{"node"}\NormalTok{, }\AttributeTok{by.y =} \StringTok{"protein\_id"}\NormalTok{)}

\NormalTok{pivotada }\OtherTok{\textless{}{-}}\NormalTok{ sensorial\_genes }\SpecialCharTok{\%\textgreater{}\%}
\NormalTok{  dplyr}\SpecialCharTok{::}\FunctionTok{select}\NormalTok{(gene\_symbol, pathway\_name) }\SpecialCharTok{\%\textgreater{}\%}
\NormalTok{  dplyr}\SpecialCharTok{::}\FunctionTok{mutate}\NormalTok{(}\AttributeTok{n =} \DecValTok{1}\NormalTok{) }\SpecialCharTok{\%\textgreater{}\%}
\NormalTok{  tidyr}\SpecialCharTok{::}\FunctionTok{pivot\_wider}\NormalTok{(}
    \AttributeTok{id\_cols =}\NormalTok{ gene\_symbol,}
    \AttributeTok{names\_from =}\NormalTok{ pathway\_name,}
    \AttributeTok{values\_from =}\NormalTok{ n,}
    \AttributeTok{values\_fn =} \FunctionTok{list}\NormalTok{(}\AttributeTok{n =}\NormalTok{ length),}
    \AttributeTok{values\_fill =} \FunctionTok{list}\NormalTok{(}\AttributeTok{n =} \DecValTok{0}\NormalTok{),}
\NormalTok{  )}

\NormalTok{source\_statements }\OtherTok{\textless{}{-}}
  \FunctionTok{colnames}\NormalTok{(pivotada)[}\DecValTok{2}\SpecialCharTok{:}\FunctionTok{length}\NormalTok{(pivotada)]}

\NormalTok{nodelist }\OtherTok{\textless{}{-}}
\NormalTok{  nodelist }\SpecialCharTok{\%\textgreater{}\%}
  \FunctionTok{left\_join}\NormalTok{(merged\_paths, }\AttributeTok{by =} \FunctionTok{c}\NormalTok{(}\StringTok{"node"} \OtherTok{=} \StringTok{"node"}\NormalTok{)) }\SpecialCharTok{\%\textgreater{}\%}
  \FunctionTok{left\_join}\NormalTok{(map\_ids, }\AttributeTok{by =} \FunctionTok{c}\NormalTok{(}\StringTok{"node"} \OtherTok{=} \StringTok{"stringId"}\NormalTok{)) }\SpecialCharTok{\%\textgreater{}\%}
  \FunctionTok{left\_join}\NormalTok{(pivotada, }\AttributeTok{by =} \FunctionTok{c}\NormalTok{(}\StringTok{"queryItem"} \OtherTok{=} \StringTok{"gene\_symbol"}\NormalTok{))}

\FunctionTok{head}\NormalTok{(nodelist)}
\end{Highlighting}
\end{Shaded}

In addition to the graph structure, we can calculate metrics such as the
number of connections (degree) of each node.

\begin{Shaded}
\begin{Highlighting}[]
\CommentTok{\# Network Metrics}
\NormalTok{connected\_nodes }\OtherTok{\textless{}{-}} \FunctionTok{rle}\NormalTok{(}\FunctionTok{sort}\NormalTok{(}\FunctionTok{c}\NormalTok{(string\_edgelist[,}\DecValTok{1}\NormalTok{], string\_edgelist[,}\DecValTok{2}\NormalTok{])))}
\NormalTok{connected\_nodes }\OtherTok{\textless{}{-}} \FunctionTok{data.frame}\NormalTok{(}\AttributeTok{count=}\NormalTok{connected\_nodes}\SpecialCharTok{$}\NormalTok{lengths, }\AttributeTok{node=}\NormalTok{connected\_nodes}\SpecialCharTok{$}\NormalTok{values)}
\NormalTok{connected\_nodes }\OtherTok{\textless{}{-}} \FunctionTok{left\_join}\NormalTok{(nodelist, connected\_nodes, }\AttributeTok{by =} \FunctionTok{c}\NormalTok{(}\StringTok{"node"} \OtherTok{=} \StringTok{"node"}\NormalTok{))}
\NormalTok{connected\_nodes }\OtherTok{\textless{}{-}}\NormalTok{ dplyr}\SpecialCharTok{::}\FunctionTok{select}\NormalTok{(connected\_nodes, queryItem, root, clade\_name, count)}

\FunctionTok{head}\NormalTok{(connected\_nodes)}
\end{Highlighting}
\end{Shaded}

\begin{Shaded}
\begin{Highlighting}[]
\CommentTok{\#nodelist |\textgreater{} }
\CommentTok{\#  vroom::vroom\_write(file = here("data/nodelist.csv"), delim = ",")}
\CommentTok{\#string\_edgelist |\textgreater{} }
\CommentTok{\#  vroom::vroom\_write(file = here("data/string\_edgelist.csv"), delim = ",")}
\CommentTok{\#merged\_paths |\textgreater{} }
\CommentTok{\#  vroom::vroom\_write(file = here("data/merged\_paths.csv"), delim = ",")}
\end{Highlighting}
\end{Shaded}

\bookmarksetup{startatroot}

\chapter{Plotting Roots}\label{plotting-roots}

\section{Import libraries}\label{import-libraries}

\begin{Shaded}
\begin{Highlighting}[]
\FunctionTok{library}\NormalTok{(ggplot2)}
\FunctionTok{library}\NormalTok{(ggraph)}
\FunctionTok{library}\NormalTok{(dplyr)}
\FunctionTok{library}\NormalTok{(tidyr)}
\FunctionTok{library}\NormalTok{(igraph)}
\FunctionTok{library}\NormalTok{(purrr)}
\FunctionTok{library}\NormalTok{(vroom)}
\FunctionTok{library}\NormalTok{(paletteer)}
\FunctionTok{library}\NormalTok{(easylayout)}
\FunctionTok{library}\NormalTok{(UpSetR)}
\FunctionTok{library}\NormalTok{(tinter)}
\FunctionTok{library}\NormalTok{(here)}
\FunctionTok{library}\NormalTok{(dplyr)}
\end{Highlighting}
\end{Shaded}

\section{Define functions}\label{define-functions}

\begin{Shaded}
\begin{Highlighting}[]
\CommentTok{\# Set colors}
\NormalTok{color\_mappings }\OtherTok{\textless{}{-}} \FunctionTok{c}\NormalTok{(}
  \StringTok{"Olfactory transduction"}   \OtherTok{=} \StringTok{"\#8dd3c7"}\NormalTok{,}
  \StringTok{"Taste transduction"}      \OtherTok{=} \StringTok{"\#72874EFF"}\NormalTok{,}
  \StringTok{"Phototransduction"}       \OtherTok{=} \StringTok{"\#fb8072"}
\NormalTok{)}

\NormalTok{subset\_graph\_by\_root }\OtherTok{\textless{}{-}}
  \ControlFlowTok{function}\NormalTok{(geneplast\_result, root\_number, graph) \{}
\NormalTok{    filtered }\OtherTok{\textless{}{-}}\NormalTok{ geneplast\_result }\SpecialCharTok{\%\textgreater{}\%}
      \FunctionTok{filter}\NormalTok{(root }\SpecialCharTok{\textgreater{}=}\NormalTok{ root\_number) }\SpecialCharTok{\%\textgreater{}\%}
      \FunctionTok{pull}\NormalTok{(node)}
    
    \FunctionTok{induced\_subgraph}\NormalTok{(graph, }\FunctionTok{which}\NormalTok{(}\FunctionTok{V}\NormalTok{(graph)}\SpecialCharTok{$}\NormalTok{name }\SpecialCharTok{\%in\%}\NormalTok{ filtered))}
\NormalTok{  \}}

\NormalTok{adjust\_color\_by\_root }\OtherTok{\textless{}{-}} \ControlFlowTok{function}\NormalTok{(geneplast\_result, root\_number, graph) \{}
\NormalTok{  filtered }\OtherTok{\textless{}{-}}\NormalTok{ geneplast\_result }\SpecialCharTok{\%\textgreater{}\%}
    \FunctionTok{filter}\NormalTok{(root }\SpecialCharTok{==}\NormalTok{ root\_number) }\SpecialCharTok{\%\textgreater{}\%}
    \FunctionTok{pull}\NormalTok{(node)}
  
  \FunctionTok{V}\NormalTok{(graph)}\SpecialCharTok{$}\NormalTok{color }\OtherTok{\textless{}{-}} \FunctionTok{ifelse}\NormalTok{(}\FunctionTok{V}\NormalTok{(graph)}\SpecialCharTok{$}\NormalTok{name }\SpecialCharTok{\%in\%}\NormalTok{ filtered, }\StringTok{"black"}\NormalTok{, }\StringTok{"gray"}\NormalTok{)}
  \FunctionTok{return}\NormalTok{(graph)}
\NormalTok{\}}

\CommentTok{\# Configure graph colors by genes incrementation}
\NormalTok{subset\_and\_adjust\_color\_by\_root }\OtherTok{\textless{}{-}} \ControlFlowTok{function}\NormalTok{(geneplast\_result, root\_number, graph) \{}
\NormalTok{  subgraph }\OtherTok{\textless{}{-}} \FunctionTok{subset\_graph\_by\_root}\NormalTok{(geneplast\_result, root\_number, graph)}
\NormalTok{  adjusted\_graph }\OtherTok{\textless{}{-}} \FunctionTok{adjust\_color\_by\_root}\NormalTok{(geneplast\_result, root\_number, subgraph)}
  \FunctionTok{return}\NormalTok{(adjusted\_graph)}
\NormalTok{\}}

\NormalTok{plot\_network }\OtherTok{\textless{}{-}} \ControlFlowTok{function}\NormalTok{(graph, title, nodelist, xlims, ylims, }\AttributeTok{legend =} \StringTok{"none"}\NormalTok{) \{}
  
  \CommentTok{\# Generate color map}
\NormalTok{  source\_statements }\OtherTok{\textless{}{-}}
    \FunctionTok{colnames}\NormalTok{(nodelist)[}\DecValTok{10}\SpecialCharTok{:}\FunctionTok{length}\NormalTok{(nodelist)]}
  
\NormalTok{  color\_mappings }\OtherTok{\textless{}{-}} \FunctionTok{c}\NormalTok{(}
    \StringTok{"Olfactory transduction"}   \OtherTok{=} \StringTok{"\#8dd3c7"}\NormalTok{,}
    \StringTok{"Taste transduction"}      \OtherTok{=} \StringTok{"\#72874EFF"}\NormalTok{,}
    \StringTok{"Phototransduction"}       \OtherTok{=} \StringTok{"\#fb8072"}
\NormalTok{  )}
  
\NormalTok{  vertices }\OtherTok{\textless{}{-}}\NormalTok{ igraph}\SpecialCharTok{::}\FunctionTok{as\_data\_frame}\NormalTok{(graph, }\StringTok{"vertices"}\NormalTok{)}
  
\NormalTok{  ggraph}\SpecialCharTok{::} \FunctionTok{ggraph}\NormalTok{(graph,}
                  \StringTok{"manual"}\NormalTok{,}
                  \AttributeTok{x =} \FunctionTok{V}\NormalTok{(graph)}\SpecialCharTok{$}\NormalTok{x,}
                  \AttributeTok{y =} \FunctionTok{V}\NormalTok{(graph)}\SpecialCharTok{$}\NormalTok{y) }\SpecialCharTok{+}
\NormalTok{    ggraph}\SpecialCharTok{::}\FunctionTok{geom\_edge\_link0}\NormalTok{(}\AttributeTok{edge\_width =} \DecValTok{1}\NormalTok{, }\AttributeTok{color =} \StringTok{"\#90909020"}\NormalTok{) }\SpecialCharTok{+}
\NormalTok{    ggraph}\SpecialCharTok{::}\FunctionTok{geom\_node\_point}\NormalTok{(ggplot2}\SpecialCharTok{::}\FunctionTok{aes}\NormalTok{(}\AttributeTok{color =} \FunctionTok{I}\NormalTok{(}\FunctionTok{V}\NormalTok{(graph)}\SpecialCharTok{$}\NormalTok{color)), }\AttributeTok{size =} \DecValTok{2}\NormalTok{) }\SpecialCharTok{+}
\NormalTok{    scatterpie}\SpecialCharTok{::}\FunctionTok{geom\_scatterpie}\NormalTok{(}
      \FunctionTok{aes}\NormalTok{(}\AttributeTok{x=}\NormalTok{x, }\AttributeTok{y=}\NormalTok{y, }\AttributeTok{r=}\DecValTok{18}\NormalTok{),}
      \AttributeTok{cols =}\NormalTok{ source\_statements,}
      \AttributeTok{data =}\NormalTok{ vertices[}\FunctionTok{rownames}\NormalTok{(vertices) }\SpecialCharTok{\%in\%} \FunctionTok{V}\NormalTok{(graph)}\SpecialCharTok{$}\NormalTok{name[}\FunctionTok{V}\NormalTok{(graph)}\SpecialCharTok{$}\NormalTok{color }\SpecialCharTok{==} \StringTok{"black"}\NormalTok{],],}
      \AttributeTok{colour =} \ConstantTok{NA}\NormalTok{,}
      \AttributeTok{pie\_scale =} \DecValTok{1}
\NormalTok{    ) }\SpecialCharTok{+}
    \FunctionTok{geom\_node\_text}\NormalTok{(}\FunctionTok{aes}\NormalTok{(}\AttributeTok{label =} \FunctionTok{ifelse}\NormalTok{(}\FunctionTok{V}\NormalTok{(graph)}\SpecialCharTok{$}\NormalTok{color }\SpecialCharTok{==} \StringTok{"black"}\NormalTok{, }\FunctionTok{V}\NormalTok{(graph)}\SpecialCharTok{$}\NormalTok{queryItem, }\ConstantTok{NA}\NormalTok{)), }
                   \AttributeTok{nudge\_x =} \DecValTok{1}\NormalTok{, }\AttributeTok{nudge\_y =} \DecValTok{1}\NormalTok{, }\AttributeTok{size =} \FloatTok{0.5}\NormalTok{, }\AttributeTok{colour =} \StringTok{"black"}\NormalTok{) }\SpecialCharTok{+}
\NormalTok{    ggplot2}\SpecialCharTok{::}\FunctionTok{scale\_fill\_manual}\NormalTok{(}\AttributeTok{values =}\NormalTok{ color\_mappings, }\AttributeTok{drop =} \ConstantTok{FALSE}\NormalTok{) }\SpecialCharTok{+}
\NormalTok{    ggplot2}\SpecialCharTok{::}\FunctionTok{coord\_fixed}\NormalTok{() }\SpecialCharTok{+}
\NormalTok{    ggplot2}\SpecialCharTok{::}\FunctionTok{scale\_x\_continuous}\NormalTok{(}\AttributeTok{limits =}\NormalTok{ xlims) }\SpecialCharTok{+}
\NormalTok{    ggplot2}\SpecialCharTok{::}\FunctionTok{scale\_y\_continuous}\NormalTok{(}\AttributeTok{limits =}\NormalTok{ ylims) }\SpecialCharTok{+}
\NormalTok{    ggplot2}\SpecialCharTok{::}\FunctionTok{theme\_void}\NormalTok{() }\SpecialCharTok{+}
\NormalTok{    ggplot2}\SpecialCharTok{::}\FunctionTok{theme}\NormalTok{(}
      \AttributeTok{legend.position =}\NormalTok{ legend,}
      \AttributeTok{legend.key.size =}\NormalTok{ ggplot2}\SpecialCharTok{::}\FunctionTok{unit}\NormalTok{(}\FloatTok{0.5}\NormalTok{, }\StringTok{\textquotesingle{}cm\textquotesingle{}}\NormalTok{),}
      \AttributeTok{legend.key.height =}\NormalTok{ ggplot2}\SpecialCharTok{::}\FunctionTok{unit}\NormalTok{(}\FloatTok{0.5}\NormalTok{, }\StringTok{\textquotesingle{}cm\textquotesingle{}}\NormalTok{),}
      \AttributeTok{legend.key.width =}\NormalTok{ ggplot2}\SpecialCharTok{::}\FunctionTok{unit}\NormalTok{(}\FloatTok{0.5}\NormalTok{, }\StringTok{\textquotesingle{}cm\textquotesingle{}}\NormalTok{),}
      \AttributeTok{legend.title =}\NormalTok{ ggplot2}\SpecialCharTok{::}\FunctionTok{element\_text}\NormalTok{(}\AttributeTok{size=}\DecValTok{6}\NormalTok{),}
      \AttributeTok{legend.text =}\NormalTok{ ggplot2}\SpecialCharTok{::}\FunctionTok{element\_text}\NormalTok{(}\AttributeTok{size=}\DecValTok{6}\NormalTok{),}
      \AttributeTok{panel.border =}\NormalTok{ ggplot2}\SpecialCharTok{::}\FunctionTok{element\_rect}\NormalTok{(}
        \AttributeTok{colour =} \StringTok{"\#161616"}\NormalTok{,}
        \AttributeTok{fill =} \ConstantTok{NA}\NormalTok{,}
        \AttributeTok{linewidth =} \DecValTok{1}
\NormalTok{      ),}
      \AttributeTok{plot.title =}\NormalTok{ ggplot2}\SpecialCharTok{::}\FunctionTok{element\_text}\NormalTok{(}\AttributeTok{size =} \DecValTok{8}\NormalTok{, }\AttributeTok{face =} \StringTok{"bold"}\NormalTok{)}
\NormalTok{    ) }\SpecialCharTok{+}
\NormalTok{    ggplot2}\SpecialCharTok{::}\FunctionTok{guides}\NormalTok{(}
      \AttributeTok{color =} \StringTok{"none"}\NormalTok{,}
      \AttributeTok{fill =} \StringTok{"none"}
\NormalTok{    ) }\SpecialCharTok{+}
\NormalTok{    ggplot2}\SpecialCharTok{::}\FunctionTok{labs}\NormalTok{(}\AttributeTok{fill =} \StringTok{"Source:"}\NormalTok{, }\AttributeTok{title =}\NormalTok{ title)}
\NormalTok{\}}
\end{Highlighting}
\end{Shaded}

\section{Loading necessary tables}\label{loading-necessary-tables}

\begin{Shaded}
\begin{Highlighting}[]
\CommentTok{\#Load data (need to save tables from first qmd)}
\NormalTok{nodelist }\OtherTok{\textless{}{-}}\NormalTok{ vroom}\SpecialCharTok{::}\FunctionTok{vroom}\NormalTok{(}\AttributeTok{file =} \FunctionTok{here}\NormalTok{(}\StringTok{"data/nodelist.csv"}\NormalTok{), }\AttributeTok{delim =} \StringTok{","}\NormalTok{)}
\NormalTok{string\_edgelist }\OtherTok{\textless{}{-}}\NormalTok{ vroom}\SpecialCharTok{::}\FunctionTok{vroom}\NormalTok{(}\AttributeTok{file =} \FunctionTok{here}\NormalTok{(}\StringTok{"data/string\_edgelist.csv"}\NormalTok{), }\AttributeTok{delim =} \StringTok{","}\NormalTok{)}
\NormalTok{merged\_paths }\OtherTok{\textless{}{-}}\NormalTok{ vroom}\SpecialCharTok{::}\FunctionTok{vroom}\NormalTok{(}\AttributeTok{file =} \FunctionTok{here}\NormalTok{(}\StringTok{"data/merged\_paths.csv"}\NormalTok{), }\AttributeTok{delim =} \StringTok{","}\NormalTok{)}
\end{Highlighting}
\end{Shaded}

\section{1. Visualization with UpSet
Plot}\label{visualization-with-upset-plot}

The \emph{UpSet Plot} is a useful tool to visualize the distribution and
concatenation of genes between different metabolic pathways. It allows
identifying how genes are shared or exclusive among the analyzed
categories.

\begin{Shaded}
\begin{Highlighting}[]
\FunctionTok{upset}\NormalTok{(dplyr}\SpecialCharTok{::}\FunctionTok{select}\NormalTok{(}\FunctionTok{as.data.frame}\NormalTok{(nodelist), }
                            \StringTok{"Olfactory transduction"}\NormalTok{,}
                            \StringTok{"Taste transduction"}\NormalTok{,}
                            \StringTok{"Phototransduction"}\NormalTok{),}
                    \AttributeTok{nsets =} \DecValTok{50}\NormalTok{, }\AttributeTok{nintersects =} \ConstantTok{NA}\NormalTok{,}
      \AttributeTok{sets.bar.color =} \FunctionTok{c}\NormalTok{(}\StringTok{"\#8dd3c7"}\NormalTok{, }\StringTok{"\#72874EFF"}\NormalTok{, }\StringTok{"\#fb8072"}\NormalTok{), }
      \AttributeTok{mainbar.y.label =} \StringTok{"Biological Process }\SpecialCharTok{\textbackslash{}n}\StringTok{Intersections"}\NormalTok{,}
      \AttributeTok{sets.x.label =} \StringTok{"Set Size"}\NormalTok{)}
\end{Highlighting}
\end{Shaded}

\includegraphics{analysis/02_plotting_roots_files/figure-pdf/unnamed-chunk-4-1.pdf}

\section{2. Visualization of the Protein-Protein Interaction
Network}\label{visualization-of-the-protein-protein-interaction-network}

The visualization of the interaction network is essential to understand
the functional connections between proteins. Here, we use the
\textbf{easylayout} package, developed by Danilo Imparato, to generate
an efficient layout. This package organizes the network nodes in x and y
coordinates, allowing a structured and clear visualization.
Subsequently, the graph will be plotted with \textbf{ggraph}.

\begin{Shaded}
\begin{Highlighting}[]
\DocumentationTok{\#\# Graph Build}
\CommentTok{\#graph \textless{}{-}}
\CommentTok{\#  graph\_from\_data\_frame(string\_edgelist, directed = FALSE, vertices = nodelist)}

\CommentTok{\#layout \textless{}{-} easylayout::easylayout(graph)}
\CommentTok{\#V(graph)$x \textless{}{-} layout[, 1]}
\CommentTok{\#V(graph)$y \textless{}{-} layout[, 2]}

\CommentTok{\#save(graph, file = "../data/graph\_layout")}
\end{Highlighting}
\end{Shaded}

\subsection{2.1. Visualization of the Ancestry of Each
Node}\label{visualization-of-the-ancestry-of-each-node}

The analysis of the ancestry of each node in the network provides an
evolutionary view of the analyzed proteins. Here, we use \textbf{ggraph}
to plot the graph with the positions previously saved by
\emph{easylayout}.

The nodes are colored according to the distance from the last common
ancestor (LCA) of the analyzed clades and the human (\emph{Human-LCA}).
The darker shade indicates older clades in relation to humans, while
light shades of blue represent newer clades, closer to the
\emph{Human-LCA}.

\begin{Shaded}
\begin{Highlighting}[]
\FunctionTok{load}\NormalTok{(}\FunctionTok{here}\NormalTok{(}\StringTok{"data/graph\_layout"}\NormalTok{))}

\FunctionTok{ggraph}\NormalTok{(graph, }\StringTok{"manual"}\NormalTok{, }\AttributeTok{x =} \FunctionTok{V}\NormalTok{(graph)}\SpecialCharTok{$}\NormalTok{x, }\AttributeTok{y =} \FunctionTok{V}\NormalTok{(graph)}\SpecialCharTok{$}\NormalTok{y) }\SpecialCharTok{+}
  \FunctionTok{geom\_edge\_link0}\NormalTok{(}\AttributeTok{color =} \StringTok{"\#90909020"}\NormalTok{) }\SpecialCharTok{+}  
  \FunctionTok{geom\_node\_point}\NormalTok{(}\FunctionTok{aes}\NormalTok{(}\AttributeTok{color =} \SpecialCharTok{{-}}\NormalTok{root), }\AttributeTok{size =} \DecValTok{2}\NormalTok{) }\SpecialCharTok{+}  
  \FunctionTok{theme\_void}\NormalTok{() }\SpecialCharTok{+} 
  \FunctionTok{theme}\NormalTok{(}\AttributeTok{legend.position =} \StringTok{"left"}\NormalTok{)}
\end{Highlighting}
\end{Shaded}

\includegraphics{analysis/02_plotting_roots_files/figure-pdf/unnamed-chunk-6-1.pdf}

\subsection{2.2. Visualization of the Protein-Protein Interaction
Network in
Human}\label{visualization-of-the-protein-protein-interaction-network-in-human}

To better understand the relationship between human proteins, we plot
the interaction network where the nodes represent the human genes
associated with their biological processes.

\subsubsection{Description of the graph
elements:}\label{description-of-the-graph-elements}

\begin{enumerate}
\def\labelenumi{\arabic{enumi}.}
\tightlist
\item
  \textbf{Nodes (Circles):} The colors of the nodes are divided
  according to the biological processes assigned to each gene. The use
  of pie charts allows the visualization of genes that participate in
  multiple processes.
\item
  \textbf{Edges (Lines):} Represent the protein interactions based on
  data from STRINGdb.
\item
  \textbf{Gene labels:} Each node is annotated with the corresponding
  gene symbol, strategically positioned for easy reading.
\end{enumerate}

\begin{Shaded}
\begin{Highlighting}[]
\DocumentationTok{\#\# Plotting Human PPI Network}
\CommentTok{\#ppi\_labaled \textless{}{-}}
\NormalTok{ggraph}\SpecialCharTok{::}\FunctionTok{ggraph}\NormalTok{(graph,}
               \StringTok{"manual"}\NormalTok{,}
               \AttributeTok{x =} \FunctionTok{V}\NormalTok{(graph)}\SpecialCharTok{$}\NormalTok{x,}
               \AttributeTok{y =} \FunctionTok{V}\NormalTok{(graph)}\SpecialCharTok{$}\NormalTok{y) }\SpecialCharTok{+}
\NormalTok{  ggraph}\SpecialCharTok{::} \FunctionTok{geom\_edge\_link0}\NormalTok{(}\AttributeTok{edge\_width =} \FloatTok{0.5}\NormalTok{, }\AttributeTok{color =} \StringTok{"\#90909020"}\NormalTok{) }\SpecialCharTok{+}
\NormalTok{  scatterpie}\SpecialCharTok{::}\FunctionTok{geom\_scatterpie}\NormalTok{(}
    \AttributeTok{cols =} \FunctionTok{colnames}\NormalTok{(nodelist[}\DecValTok{10}\SpecialCharTok{:}\DecValTok{12}\NormalTok{]),}
    \AttributeTok{data =}\NormalTok{ igraph}\SpecialCharTok{::}\FunctionTok{as\_data\_frame}\NormalTok{(graph, }\StringTok{"vertices"}\NormalTok{),}
    \AttributeTok{colour =} \ConstantTok{NA}\NormalTok{,}
    \AttributeTok{pie\_scale =} \FloatTok{0.40}
\NormalTok{  ) }\SpecialCharTok{+}
  \FunctionTok{geom\_node\_text}\NormalTok{(}\FunctionTok{aes}\NormalTok{(}\AttributeTok{label =}\NormalTok{ nodelist}\SpecialCharTok{$}\NormalTok{queryItem), }\AttributeTok{colour =} \StringTok{"black"}\NormalTok{, }\AttributeTok{nudge\_x =} \FloatTok{0.8}\NormalTok{, }\AttributeTok{nudge\_y =} \FloatTok{0.8}\NormalTok{, }\AttributeTok{size =} \DecValTok{2}\NormalTok{) }\SpecialCharTok{+}
\NormalTok{  ggplot2}\SpecialCharTok{::}\FunctionTok{scale\_fill\_manual}\NormalTok{(}\AttributeTok{values =}\NormalTok{ color\_mappings, }\AttributeTok{drop =} \ConstantTok{FALSE}\NormalTok{)}
\end{Highlighting}
\end{Shaded}

\includegraphics{analysis/02_plotting_roots_files/figure-pdf/unnamed-chunk-7-1.pdf}

\begin{Shaded}
\begin{Highlighting}[]
\CommentTok{\#ppi \textless{}{-} }
\NormalTok{  ggraph}\SpecialCharTok{::}\FunctionTok{ggraph}\NormalTok{(graph,}
               \StringTok{"manual"}\NormalTok{,}
               \AttributeTok{x =} \FunctionTok{V}\NormalTok{(graph)}\SpecialCharTok{$}\NormalTok{x,}
               \AttributeTok{y =} \FunctionTok{V}\NormalTok{(graph)}\SpecialCharTok{$}\NormalTok{y) }\SpecialCharTok{+}
\NormalTok{  ggraph}\SpecialCharTok{::} \FunctionTok{geom\_edge\_link0}\NormalTok{(}\AttributeTok{edge\_width =} \FloatTok{0.5}\NormalTok{, }\AttributeTok{color =} \StringTok{"\#90909020"}\NormalTok{) }\SpecialCharTok{+}
\NormalTok{  scatterpie}\SpecialCharTok{::}\FunctionTok{geom\_scatterpie}\NormalTok{(}
    \AttributeTok{cols =} \FunctionTok{colnames}\NormalTok{(nodelist[}\DecValTok{10}\SpecialCharTok{:}\DecValTok{12}\NormalTok{]),}
    \AttributeTok{data =}\NormalTok{ igraph}\SpecialCharTok{::}\FunctionTok{as\_data\_frame}\NormalTok{(graph, }\StringTok{"vertices"}\NormalTok{),}
    \AttributeTok{colour =} \ConstantTok{NA}\NormalTok{,}
    \AttributeTok{pie\_scale =} \FloatTok{0.40}
\NormalTok{  ) }\SpecialCharTok{+}
\NormalTok{  ggplot2}\SpecialCharTok{::}\FunctionTok{scale\_fill\_manual}\NormalTok{(}\AttributeTok{values =}\NormalTok{ color\_mappings, }\AttributeTok{drop =} \ConstantTok{FALSE}\NormalTok{)}
\end{Highlighting}
\end{Shaded}

\includegraphics{analysis/02_plotting_roots_files/figure-pdf/unnamed-chunk-7-2.pdf}

\subsection{2.3. Visualization of the Protein-Protein Interaction
Network in Each
Clade}\label{visualization-of-the-protein-protein-interaction-network-in-each-clade}

In this section, we visualize the genes that are statistically rooted in
each clade. The arrangement of the genes allows us to observe the
increment of orthologous genes as a function of the complexity and
antiquity of the biological system.

\subsubsection{Visualization features:}\label{visualization-features}

\begin{enumerate}
\def\labelenumi{\arabic{enumi}.}
\tightlist
\item
  \textbf{Evolution of the graphs:} The graphs are organized from left
  to right and from top to bottom, allowing the analysis of the
  evolutionary progression.
\item
  \textbf{Node coloring:} The color of the nodes indicates the level of
  ancestry, as previously highlighted, where darker tones represent
  older clades and lighter tones indicate evolutionary proximity to
  humans.
\item
  \textbf{Organisms of interest:} In addition to visualizing all clades,
  it is possible to generate graphs focused only on certain groups, such
  as \emph{Metamonada}, \emph{Choanoflagellata}, \emph{Cephalochordata},
  and \emph{Amphibia}.
\end{enumerate}

With these visualizations, it is possible to identify patterns of gene
evolution in different clades and perform detailed comparisons with
organisms of specific interest.

\begin{Shaded}
\begin{Highlighting}[]
\NormalTok{geneplast\_roots }\OtherTok{\textless{}{-}}\NormalTok{ merged\_paths[}\FunctionTok{order}\NormalTok{(merged\_paths}\SpecialCharTok{$}\NormalTok{root), ]}
  

\NormalTok{buffer }\OtherTok{\textless{}{-}} \FunctionTok{c}\NormalTok{(}\SpecialCharTok{{-}}\DecValTok{50}\NormalTok{, }\DecValTok{50}\NormalTok{)}
\NormalTok{xlims }\OtherTok{\textless{}{-}} \FunctionTok{ceiling}\NormalTok{(}\FunctionTok{range}\NormalTok{(}\FunctionTok{V}\NormalTok{(graph)}\SpecialCharTok{$}\NormalTok{x)) }\SpecialCharTok{+}\NormalTok{ buffer}
\NormalTok{ylims }\OtherTok{\textless{}{-}} \FunctionTok{ceiling}\NormalTok{(}\FunctionTok{range}\NormalTok{(}\FunctionTok{V}\NormalTok{(graph)}\SpecialCharTok{$}\NormalTok{y)) }\SpecialCharTok{+}\NormalTok{ buffer}

\NormalTok{roots }\OtherTok{\textless{}{-}} \FunctionTok{unique}\NormalTok{(geneplast\_roots}\SpecialCharTok{$}\NormalTok{root) }\SpecialCharTok{\%\textgreater{}\%}
  \FunctionTok{set\_names}\NormalTok{(}\FunctionTok{unique}\NormalTok{(geneplast\_roots}\SpecialCharTok{$}\NormalTok{clade\_name))}

\CommentTok{\# Subset graphs by LCAs}
\NormalTok{subsets }\OtherTok{\textless{}{-}}
  \FunctionTok{map}\NormalTok{(roots, }\SpecialCharTok{\textasciitilde{}} \FunctionTok{subset\_and\_adjust\_color\_by\_root}\NormalTok{(geneplast\_roots, .x, graph))}

\CommentTok{\# Plot titles}
\NormalTok{titles }\OtherTok{\textless{}{-}} \FunctionTok{names}\NormalTok{(roots)}

\NormalTok{plots }\OtherTok{\textless{}{-}}
  \FunctionTok{map2}\NormalTok{(}
\NormalTok{    subsets,}
\NormalTok{    titles,}
\NormalTok{    plot\_network,}
    \AttributeTok{nodelist =}\NormalTok{ nodelist,}
    \AttributeTok{xlims =}\NormalTok{ xlims,}
    \AttributeTok{ylims =}\NormalTok{ ylims,}
    \AttributeTok{legend =} \StringTok{"right"}
\NormalTok{  ) }\SpecialCharTok{\%\textgreater{}\%}
  \FunctionTok{discard}\NormalTok{(is.null)}

\CommentTok{\#net\_all\_roots \textless{}{-}}
\NormalTok{patchwork}\SpecialCharTok{::}\FunctionTok{wrap\_plots}\NormalTok{(}
  \FunctionTok{rev}\NormalTok{(plots),}
  \AttributeTok{nrow =} \DecValTok{4}\NormalTok{,}
  \AttributeTok{ncol =} \DecValTok{4}
\NormalTok{)}
\end{Highlighting}
\end{Shaded}

\includegraphics{analysis/02_plotting_roots_files/figure-pdf/unnamed-chunk-8-1.pdf}

\begin{Shaded}
\begin{Highlighting}[]
\CommentTok{\#ggsave(file = "../data/network\_rooting.svg", plot=net\_all\_roots, width=10, height=8)}
\end{Highlighting}
\end{Shaded}

\begin{Shaded}
\begin{Highlighting}[]
\NormalTok{patchwork}\SpecialCharTok{::}\FunctionTok{wrap\_plots}\NormalTok{(}
\NormalTok{  plots}\SpecialCharTok{$}\NormalTok{Metamonada, plots}\SpecialCharTok{$}\NormalTok{Choanoflagellata, plots}\SpecialCharTok{$}\NormalTok{Cephalochordata, plots}\SpecialCharTok{$}\NormalTok{Amphibia,}
  \AttributeTok{ncol =} \DecValTok{4}
\NormalTok{)}
\end{Highlighting}
\end{Shaded}

\includegraphics{analysis/02_plotting_roots_files/figure-pdf/unnamed-chunk-9-1.pdf}

\bookmarksetup{startatroot}

\chapter{Plotting Abundance}\label{plotting-abundance}

\section{Importing Packages}\label{importing-packages-1}

\begin{Shaded}
\begin{Highlighting}[]
\FunctionTok{library}\NormalTok{(here)}
\FunctionTok{library}\NormalTok{(readr)}
\FunctionTok{library}\NormalTok{(magrittr)}
\FunctionTok{library}\NormalTok{(ggplot2)}
\FunctionTok{library}\NormalTok{(hrbrthemes)}
\FunctionTok{library}\NormalTok{(tinter)}
\FunctionTok{library}\NormalTok{(dplyr)}
\FunctionTok{library}\NormalTok{(tidyr)}
\FunctionTok{library}\NormalTok{(AnnotationHub)}
\end{Highlighting}
\end{Shaded}

\section{Defining functions}\label{defining-functions}

These functions are intended to organize the annotation of the vertices
of our nodelist and calculate the cumulative number of genes per clade
and per biological process.

\begin{Shaded}
\begin{Highlighting}[]
\NormalTok{calculate\_cumulative\_genes }\OtherTok{\textless{}{-}} \ControlFlowTok{function}\NormalTok{(nodelist) \{}
  
    \CommentTok{\# Get all possible categories of clade\_name}
\NormalTok{  all\_clades }\OtherTok{\textless{}{-}}\NormalTok{ node\_annotation }\SpecialCharTok{\%\textgreater{}\%}
    \FunctionTok{arrange}\NormalTok{(}\FunctionTok{desc}\NormalTok{(root)) }\SpecialCharTok{\%\textgreater{}\%}
\NormalTok{    dplyr}\SpecialCharTok{::} \FunctionTok{select}\NormalTok{(clade\_name) }\SpecialCharTok{\%\textgreater{}\%}
    \FunctionTok{unique}\NormalTok{()}
  
  \CommentTok{\# Define the columns of interest}
\NormalTok{  process\_columns }\OtherTok{\textless{}{-}} \FunctionTok{c}\NormalTok{(}\StringTok{"queryItem"}\NormalTok{, }\StringTok{"root"}\NormalTok{, }\StringTok{"clade\_name"}\NormalTok{, }
                       \StringTok{"Olfactory transduction"}\NormalTok{, }
                       \StringTok{"Taste transduction"}\NormalTok{,           }
                       \StringTok{"Phototransduction"}\NormalTok{)}
  
  \CommentTok{\# Calculate the cumulative sum grouping by clade\_name}
\NormalTok{  cumulative\_genes }\OtherTok{\textless{}{-}}\NormalTok{ nodelist }\SpecialCharTok{\%\textgreater{}\%}
    \FunctionTok{arrange}\NormalTok{(}\FunctionTok{desc}\NormalTok{(root)) }\SpecialCharTok{\%\textgreater{}\%}
\NormalTok{    dplyr}\SpecialCharTok{::}\FunctionTok{select}\NormalTok{(}\FunctionTok{all\_of}\NormalTok{(process\_columns)) }\SpecialCharTok{\%\textgreater{}\%}
    \FunctionTok{group\_by}\NormalTok{(clade\_name, root) }\SpecialCharTok{\%\textgreater{}\%}
    \FunctionTok{summarise}\NormalTok{(}\AttributeTok{count\_genes =} \FunctionTok{n}\NormalTok{(), }\AttributeTok{.groups =} \StringTok{"drop"}\NormalTok{) }\SpecialCharTok{\%\textgreater{}\%}
    \FunctionTok{arrange}\NormalTok{(}\FunctionTok{desc}\NormalTok{(root)) }\SpecialCharTok{\%\textgreater{}\%}
    \FunctionTok{mutate}\NormalTok{(}\AttributeTok{cumulative\_sum =} \FunctionTok{cumsum}\NormalTok{(count\_genes)) }\SpecialCharTok{\%\textgreater{}\%}
    \FunctionTok{right\_join}\NormalTok{(all\_clades, }\AttributeTok{by =} \StringTok{"clade\_name"}\NormalTok{) }\SpecialCharTok{\%\textgreater{}\%}
    \FunctionTok{fill}\NormalTok{(cumulative\_sum, }\AttributeTok{.direction =} \StringTok{"down"}\NormalTok{)}
  
  \FunctionTok{return}\NormalTok{(cumulative\_genes)}
\NormalTok{\}}

\NormalTok{calculate\_cumulative\_bp }\OtherTok{\textless{}{-}} \ControlFlowTok{function}\NormalTok{(nodelist) \{}
    
  \CommentTok{\# Get all possible categories of clade\_name}
\NormalTok{  all\_clades }\OtherTok{\textless{}{-}}\NormalTok{ node\_annotation }\SpecialCharTok{\%\textgreater{}\%}
    \FunctionTok{arrange}\NormalTok{(}\FunctionTok{desc}\NormalTok{(root)) }\SpecialCharTok{\%\textgreater{}\%}
\NormalTok{    dplyr}\SpecialCharTok{::} \FunctionTok{select}\NormalTok{(clade\_name) }\SpecialCharTok{\%\textgreater{}\%}
    \FunctionTok{unique}\NormalTok{()}
  
   \CommentTok{\# Define the columns of interest}
\NormalTok{  process\_columns }\OtherTok{\textless{}{-}} \FunctionTok{c}\NormalTok{(}\StringTok{"queryItem"}\NormalTok{, }\StringTok{"root"}\NormalTok{, }\StringTok{"clade\_name"}\NormalTok{, }
                       \StringTok{"Olfactory transduction"}\NormalTok{, }
                       \StringTok{"Taste transduction"}\NormalTok{,           }
                       \StringTok{"Phototransduction"}\NormalTok{)}
  
  \CommentTok{\# Calculate the cumulative sum for each biological process}
\NormalTok{  cumulative\_bp }\OtherTok{\textless{}{-}}\NormalTok{ nodelist }\SpecialCharTok{\%\textgreater{}\%}
\NormalTok{    dplyr}\SpecialCharTok{::}\FunctionTok{select}\NormalTok{(}\FunctionTok{all\_of}\NormalTok{(process\_columns)) }\SpecialCharTok{\%\textgreater{}\%}
    \FunctionTok{distinct}\NormalTok{(root, queryItem, }\AttributeTok{.keep\_all =} \ConstantTok{TRUE}\NormalTok{) }\SpecialCharTok{\%\textgreater{}\%}
    \FunctionTok{mutate}\NormalTok{(}\FunctionTok{across}\NormalTok{(}\FunctionTok{all\_of}\NormalTok{(process\_columns[}\SpecialCharTok{{-}}\FunctionTok{c}\NormalTok{(}\DecValTok{1}\SpecialCharTok{:}\DecValTok{3}\NormalTok{)]), }\SpecialCharTok{\textasciitilde{}} \FunctionTok{as.numeric}\NormalTok{(.))) }\SpecialCharTok{\%\textgreater{}\%}
    \FunctionTok{group\_by}\NormalTok{(root, clade\_name) }\SpecialCharTok{\%\textgreater{}\%}
    \FunctionTok{summarise}\NormalTok{(}\FunctionTok{across}\NormalTok{(}\FunctionTok{all\_of}\NormalTok{(process\_columns[}\SpecialCharTok{{-}}\FunctionTok{c}\NormalTok{(}\DecValTok{1}\SpecialCharTok{:}\DecValTok{3}\NormalTok{)]), }
                     \SpecialCharTok{\textasciitilde{}} \FunctionTok{sum}\NormalTok{(. , }\AttributeTok{na.rm =} \ConstantTok{TRUE}\NormalTok{)),}
              \AttributeTok{.groups =} \StringTok{"drop"}\NormalTok{) }\SpecialCharTok{\%\textgreater{}\%}
    \FunctionTok{arrange}\NormalTok{(}\FunctionTok{desc}\NormalTok{(root)) }\SpecialCharTok{\%\textgreater{}\%}
    \FunctionTok{mutate}\NormalTok{(}\FunctionTok{across}\NormalTok{(}\FunctionTok{all\_of}\NormalTok{(process\_columns[}\SpecialCharTok{{-}}\FunctionTok{c}\NormalTok{(}\DecValTok{1}\SpecialCharTok{:}\DecValTok{3}\NormalTok{)]), }\SpecialCharTok{\textasciitilde{}} \FunctionTok{cumsum}\NormalTok{(.))) }\SpecialCharTok{\%\textgreater{}\%}
    \FunctionTok{right\_join}\NormalTok{(all\_clades, }\AttributeTok{by =} \StringTok{"clade\_name"}\NormalTok{) }\SpecialCharTok{\%\textgreater{}\%}
    \FunctionTok{fill}\NormalTok{(}\FunctionTok{everything}\NormalTok{(), }\AttributeTok{.direction =} \StringTok{"down"}\NormalTok{)}
  
  \FunctionTok{return}\NormalTok{(cumulative\_bp)}
\NormalTok{\}}
\end{Highlighting}
\end{Shaded}

\section{Defining aesthetic parameters for
ggplot}\label{defining-aesthetic-parameters-for-ggplot}

Assigning default colors for the metabolic pathways in the analysis and
defining the other aesthetic standards for plotting.

\begin{Shaded}
\begin{Highlighting}[]
\CommentTok{\# Plotting colors and labels}
\NormalTok{annotation\_colors }\OtherTok{\textless{}{-}} \FunctionTok{c}\NormalTok{(}
  \StringTok{"Olfactory transduction"}   \OtherTok{=} \StringTok{"\#8dd3c7"}\NormalTok{,}
  \StringTok{"Taste transduction"}      \OtherTok{=} \StringTok{"\#72874EFF"}\NormalTok{,}
  \StringTok{"Phototransduction"}       \OtherTok{=} \StringTok{"\#fb8072"}
\NormalTok{)}

\NormalTok{annotation\_labels }\OtherTok{\textless{}{-}} \FunctionTok{c}\NormalTok{(}
  \StringTok{"Olfactory transduction"}   \OtherTok{=} \StringTok{"Olfactory transduction"}\NormalTok{,}
  \StringTok{"Taste transduction"}      \OtherTok{=} \StringTok{"Taste transduction"}\NormalTok{,}
  \StringTok{"Phototransduction"}       \OtherTok{=} \StringTok{"Phototransduction"}
\NormalTok{)}

\CommentTok{\# This vertical line indicates the first metazoan (Amphimedon queenslandica / Ctenophora)}
\NormalTok{choanoflagellata\_line }\OtherTok{\textless{}{-}} \FunctionTok{geom\_vline}\NormalTok{(}
  \AttributeTok{xintercept =} \StringTok{"Sphaeroforma arctica"}\NormalTok{,}
  \AttributeTok{color      =} \StringTok{"\#FF0000"}\NormalTok{,}
  \AttributeTok{linetype   =} \StringTok{"11"}\NormalTok{,}
  \AttributeTok{alpha      =} \DecValTok{1}\NormalTok{,}
  \AttributeTok{linewidth  =} \FloatTok{0.25}
\NormalTok{)}

\CommentTok{\# Plotting}
\NormalTok{theme\_main }\OtherTok{\textless{}{-}} \FunctionTok{theme}\NormalTok{(}
  \AttributeTok{panel.spacing      =} \FunctionTok{unit}\NormalTok{(}\FloatTok{2.5}\NormalTok{, }\StringTok{"pt"}\NormalTok{),}
  \AttributeTok{strip.background   =} \FunctionTok{element\_blank}\NormalTok{(),}
  \AttributeTok{panel.grid.major.x =} \FunctionTok{element\_blank}\NormalTok{(),}
  \AttributeTok{panel.grid.major.y =} \FunctionTok{element\_line}\NormalTok{(}\AttributeTok{linewidth =} \FloatTok{0.25}\NormalTok{, }\AttributeTok{linetype =} \StringTok{"dotted"}\NormalTok{, }\AttributeTok{color =} \StringTok{"\#E0E0E0"}\NormalTok{),}
  \AttributeTok{strip.text.x       =} \FunctionTok{element\_text}\NormalTok{(}\AttributeTok{size =} \DecValTok{9}\NormalTok{, }\AttributeTok{angle =} \DecValTok{90}\NormalTok{, }\AttributeTok{hjust =} \DecValTok{0}\NormalTok{, }\AttributeTok{vjust =} \FloatTok{0.5}\NormalTok{, }\AttributeTok{color =} \StringTok{"\#757575"}\NormalTok{),}
  \AttributeTok{strip.text.y       =} \FunctionTok{element\_text}\NormalTok{(}\AttributeTok{size =} \DecValTok{10}\NormalTok{, }\AttributeTok{angle =} \DecValTok{0}\NormalTok{, }\AttributeTok{hjust =} \DecValTok{0}\NormalTok{, }\AttributeTok{vjust =} \FloatTok{0.5}\NormalTok{, }\AttributeTok{color =} \StringTok{"\#757575"}\NormalTok{),}
  \AttributeTok{axis.title         =} \FunctionTok{element\_text}\NormalTok{(}\AttributeTok{size =} \DecValTok{15}\NormalTok{, }\AttributeTok{color =} \StringTok{"\#424242"}\NormalTok{),}
  \AttributeTok{axis.ticks.x       =} \FunctionTok{element\_blank}\NormalTok{(),}
  \AttributeTok{axis.text.x        =} \FunctionTok{element\_blank}\NormalTok{(),}
  \AttributeTok{axis.text.y        =} \FunctionTok{element\_text}\NormalTok{(}\AttributeTok{size =} \FloatTok{5.5}\NormalTok{),}
  \AttributeTok{legend.position    =} \StringTok{"none"}
\NormalTok{)}

\NormalTok{theme\_supplementary }\OtherTok{\textless{}{-}} \FunctionTok{theme}\NormalTok{(}
  \AttributeTok{panel.grid.major.x =} \FunctionTok{element\_line}\NormalTok{(}\AttributeTok{color =} \StringTok{"\#E0E0E0"}\NormalTok{, }\AttributeTok{linewidth =} \FloatTok{0.25}\NormalTok{, }\AttributeTok{linetype =} \StringTok{"dotted"}\NormalTok{),}
  \AttributeTok{panel.grid.major.y =} \FunctionTok{element\_blank}\NormalTok{(),}
  \AttributeTok{strip.text.y       =} \FunctionTok{element\_text}\NormalTok{(}\AttributeTok{size =} \DecValTok{7}\NormalTok{, }\AttributeTok{angle =} \DecValTok{0}\NormalTok{, }\AttributeTok{hjust =} \DecValTok{0}\NormalTok{, }\AttributeTok{vjust =} \FloatTok{0.5}\NormalTok{, }\AttributeTok{color =} \StringTok{"\#757575"}\NormalTok{),}
  \AttributeTok{strip.text.x       =} \FunctionTok{element\_text}\NormalTok{(}\AttributeTok{size =} \DecValTok{7}\NormalTok{, }\AttributeTok{angle =} \DecValTok{90}\NormalTok{, }\AttributeTok{hjust =} \DecValTok{0}\NormalTok{, }\AttributeTok{vjust =} \FloatTok{0.5}\NormalTok{, }\AttributeTok{color =} \StringTok{"\#757575"}\NormalTok{),}
  \AttributeTok{axis.title         =} \FunctionTok{element\_text}\NormalTok{(}\AttributeTok{size =} \DecValTok{12}\NormalTok{, }\AttributeTok{color =} \StringTok{"\#424242"}\NormalTok{),}
  \AttributeTok{axis.ticks         =} \FunctionTok{element\_line}\NormalTok{(}\AttributeTok{colour =} \StringTok{"grey20"}\NormalTok{),}
  \AttributeTok{axis.text.y        =} \FunctionTok{element\_text}\NormalTok{(}\AttributeTok{size =} \DecValTok{6}\NormalTok{, }\AttributeTok{angle =} \DecValTok{0}\NormalTok{, }\AttributeTok{hjust =} \DecValTok{1}\NormalTok{, }\AttributeTok{vjust =} \FloatTok{0.5}\NormalTok{, }\AttributeTok{color =} \StringTok{"\#757575"}\NormalTok{),}
  \AttributeTok{axis.text.x        =} \FunctionTok{element\_text}\NormalTok{(}\AttributeTok{size =} \DecValTok{6}\NormalTok{)}
\NormalTok{)}

\NormalTok{theme\_average }\OtherTok{\textless{}{-}} \FunctionTok{theme}\NormalTok{(}
  \AttributeTok{panel.spacing      =} \FunctionTok{unit}\NormalTok{(}\DecValTok{1}\NormalTok{, }\StringTok{"pt"}\NormalTok{),}
  \AttributeTok{axis.title         =} \FunctionTok{element\_text}\NormalTok{(}\AttributeTok{color =} \StringTok{"\#424242"}\NormalTok{),}
  \AttributeTok{axis.text          =} \FunctionTok{element\_text}\NormalTok{(}\AttributeTok{color =} \StringTok{"\#757575"}\NormalTok{),}
  \AttributeTok{axis.text.x        =} \FunctionTok{element\_text}\NormalTok{(}\AttributeTok{size =} \DecValTok{7}\NormalTok{, }\AttributeTok{angle =} \SpecialCharTok{{-}}\DecValTok{45}\NormalTok{, }\AttributeTok{vjust =} \DecValTok{0}\NormalTok{, }\AttributeTok{hjust =} \DecValTok{0}\NormalTok{),}
  \AttributeTok{axis.text.y        =} \FunctionTok{element\_text}\NormalTok{(}\AttributeTok{size =} \DecValTok{5}\NormalTok{),}
  \AttributeTok{strip.background   =} \FunctionTok{element\_blank}\NormalTok{(),}
  \AttributeTok{strip.text         =} \FunctionTok{element\_text}\NormalTok{(}\AttributeTok{color =} \StringTok{"\#757575"}\NormalTok{),}
  \AttributeTok{strip.text.y       =} \FunctionTok{element\_text}\NormalTok{(}\AttributeTok{angle =} \DecValTok{0}\NormalTok{, }\AttributeTok{hjust =} \DecValTok{0}\NormalTok{, }\AttributeTok{vjust =} \FloatTok{0.5}\NormalTok{)}
\NormalTok{)}

\NormalTok{theme\_big }\OtherTok{\textless{}{-}} \FunctionTok{theme}\NormalTok{(}
  \AttributeTok{panel.spacing      =} \FunctionTok{unit}\NormalTok{(}\FloatTok{0.5}\NormalTok{, }\StringTok{"pt"}\NormalTok{),}
  \AttributeTok{panel.grid.major.x =} \FunctionTok{element\_line}\NormalTok{(}\AttributeTok{linewidth =} \FloatTok{0.1}\NormalTok{, }\AttributeTok{linetype =} \StringTok{"dashed"}\NormalTok{),}
  \AttributeTok{panel.grid.major.y =} \FunctionTok{element\_blank}\NormalTok{(),}
  \AttributeTok{strip.background   =} \FunctionTok{element\_blank}\NormalTok{(),}
  \AttributeTok{strip.text.x       =} \FunctionTok{element\_text}\NormalTok{(}\AttributeTok{size =} \DecValTok{8}\NormalTok{, }\AttributeTok{angle =} \DecValTok{90}\NormalTok{, }\AttributeTok{hjust =} \FloatTok{0.5}\NormalTok{, }\AttributeTok{vjust =} \DecValTok{0}\NormalTok{),}
  \AttributeTok{strip.text.y       =} \FunctionTok{element\_text}\NormalTok{(}\AttributeTok{size =} \DecValTok{8}\NormalTok{, }\AttributeTok{angle =} \DecValTok{0}\NormalTok{, }\AttributeTok{hjust =} \DecValTok{0}\NormalTok{, }\AttributeTok{vjust =} \FloatTok{0.5}\NormalTok{),}
  \AttributeTok{axis.text.x        =} \FunctionTok{element\_text}\NormalTok{(}\AttributeTok{size =} \DecValTok{6}\NormalTok{, }\AttributeTok{angle =} \DecValTok{90}\NormalTok{, }\AttributeTok{vjust =} \DecValTok{0}\NormalTok{, }\AttributeTok{hjust =} \DecValTok{0}\NormalTok{),}
  \AttributeTok{axis.text.y        =} \FunctionTok{element\_text}\NormalTok{(}\AttributeTok{size =} \FloatTok{4.5}\NormalTok{),}
  \AttributeTok{axis.ticks         =} \FunctionTok{element\_line}\NormalTok{(}\AttributeTok{size =} \FloatTok{0.1}\NormalTok{)}
\NormalTok{)}

\NormalTok{tick\_function }\OtherTok{\textless{}{-}} \ControlFlowTok{function}\NormalTok{(x) \{}
  \FunctionTok{seq}\NormalTok{(x[}\DecValTok{2}\NormalTok{], }\DecValTok{0}\NormalTok{, }\AttributeTok{length.out =} \DecValTok{3}\NormalTok{) }\SpecialCharTok{\%\textgreater{}\%} \FunctionTok{head}\NormalTok{(}\SpecialCharTok{{-}}\DecValTok{1}\NormalTok{) }\SpecialCharTok{\%\textgreater{}\%} \FunctionTok{tail}\NormalTok{(}\SpecialCharTok{{-}}\DecValTok{1}\NormalTok{) }\SpecialCharTok{\%\textgreater{}\%}\NormalTok{ \{ }\FunctionTok{ceiling}\NormalTok{(.}\SpecialCharTok{/}\DecValTok{5}\NormalTok{)}\SpecialCharTok{*}\DecValTok{5}\NormalTok{ \}}
\NormalTok{\}}
\end{Highlighting}
\end{Shaded}

\section{Load necessary tables}\label{load-necessary-tables}

To proceed with the analysis, it is necessary to load a table containing
the \emph{taxid} information for each species. This information will be
used later to map the \emph{taxid} of the species with the \emph{taxid}
of each COG.

\subsection{Table Description}\label{table-description}

\begin{itemize}
\tightlist
\item
  \textbf{File Name}: \texttt{string\_eukaryotes.rda}
\item
  \textbf{Content}: Taxonomic information of eukaryotic species.
\end{itemize}

The table loads the following main fields: - \textbf{TaxID}
(\emph{taxonomic identifier}): A unique identifier associated with each
species. - Species name

\begin{Shaded}
\begin{Highlighting}[]
\CommentTok{\# Query Phylotree and OG data}
\NormalTok{ah }\OtherTok{\textless{}{-}} \FunctionTok{AnnotationHub}\NormalTok{()}
\NormalTok{meta }\OtherTok{\textless{}{-}} \FunctionTok{query}\NormalTok{(ah, }\StringTok{"geneplast"}\NormalTok{)}
\FunctionTok{load}\NormalTok{(meta[[}\StringTok{"AH83116"}\NormalTok{]])}

\CommentTok{\# }\AlertTok{TODO}\CommentTok{: save tables from previous qmds}
\NormalTok{nodelist }\OtherTok{\textless{}{-}}\NormalTok{ vroom}\SpecialCharTok{::}\FunctionTok{vroom}\NormalTok{(}\AttributeTok{file =} \FunctionTok{here}\NormalTok{(}\StringTok{"data/nodelist.csv"}\NormalTok{), }\AttributeTok{delim =} \StringTok{","}\NormalTok{)}
\NormalTok{gene\_cogs }\OtherTok{\textless{}{-}}\NormalTok{  vroom}\SpecialCharTok{::}\FunctionTok{vroom}\NormalTok{(}\AttributeTok{file =} \FunctionTok{here}\NormalTok{(}\StringTok{"data/gene\_cogs.csv"}\NormalTok{), }\AttributeTok{delim =} \StringTok{","}\NormalTok{)}
\NormalTok{sensorial\_genes }\OtherTok{\textless{}{-}} \FunctionTok{read.csv}\NormalTok{(}\FunctionTok{here}\NormalTok{(}\StringTok{"data/sensorial\_genes.csv"}\NormalTok{))}
\NormalTok{map\_ids }\OtherTok{\textless{}{-}}\NormalTok{ vroom}\SpecialCharTok{::}\FunctionTok{vroom}\NormalTok{(}\FunctionTok{here}\NormalTok{(}\StringTok{"data/map\_ids.csv"}\NormalTok{))}
\NormalTok{groot\_df }\OtherTok{\textless{}{-}}\NormalTok{ vroom}\SpecialCharTok{::}\FunctionTok{vroom}\NormalTok{(}\FunctionTok{here}\NormalTok{(}\StringTok{"data/groot\_df.csv"}\NormalTok{))}

\NormalTok{ogr }\OtherTok{\textless{}{-}} \FunctionTok{readRDS}\NormalTok{(}\FunctionTok{here}\NormalTok{(}\StringTok{"data/ogr.RData"}\NormalTok{))}

\FunctionTok{load}\NormalTok{(}\FunctionTok{here}\NormalTok{(}\StringTok{"data/string\_eukaryotes.rda"}\NormalTok{))}
\FunctionTok{head}\NormalTok{(string\_eukaryotes)}
\end{Highlighting}
\end{Shaded}

\section{Visualization: Rooting
Pattern}\label{visualization-rooting-pattern}

This visualization allows analyzing the cumulative pattern of gene
emergence over evolutionary time. In addition, it also shows the growth
of biological processes separated by each metabolic pathway.

\subsection{Graph Characteristics}\label{graph-characteristics}

\begin{enumerate}
\def\labelenumi{\arabic{enumi}.}
\tightlist
\item
  \textbf{Bar Chart (Cumulative Pattern)}:

  \begin{itemize}
  \tightlist
  \item
    Represents the cumulative sum of genes associated with each clade.
  \item
    Each bar displays the total accumulated genes in a given clade,
    allowing the identification of the diversification pattern.
  \end{itemize}
\item
  \textbf{Combined Chart (Bars and Lines)}:

  \begin{itemize}
  \tightlist
  \item
    The bars indicate the cumulative sum of genes per clade.
  \item
    The lines represent the growth of different biological processes
    (\emph{Process}), separated by metabolic pathways, allowing the
    comparison between the cumulative emergence of genes and the
    development of biological functions.
  \end{itemize}
\end{enumerate}

\subsection{Interpretation}\label{interpretation}

\begin{itemize}
\tightlist
\item
  \textbf{Cumulative Pattern}: The bar chart shows how genes have
  emerged over time, cumulatively. This view helps to identify moments
  of greater genetic diversification.
\item
  \textbf{Comparison by Biological Processes}: The combined chart allows
  observing which biological processes stand out at different moments of
  evolution and how they are related to the emergence of new genes.
\end{itemize}

\begin{Shaded}
\begin{Highlighting}[]
\CommentTok{\# Mapping roots and proteins info}
\NormalTok{node\_annotation }\OtherTok{\textless{}{-}}\NormalTok{ nodelist }\SpecialCharTok{\%\textgreater{}\%}
  \FunctionTok{inner\_join}\NormalTok{(gene\_cogs, }\AttributeTok{by =} \FunctionTok{c}\NormalTok{(}\StringTok{"node"} \OtherTok{=} \StringTok{"protein\_id"}\NormalTok{, }\StringTok{"cog\_id"}\NormalTok{)) }\SpecialCharTok{\%\textgreater{}\%}
  \FunctionTok{inner\_join}\NormalTok{(sensorial\_genes, }\AttributeTok{by =} \FunctionTok{c}\NormalTok{(}\StringTok{"queryItem"} \OtherTok{=} \StringTok{"gene\_symbol"}\NormalTok{)) }\SpecialCharTok{\%\textgreater{}\%}
  \FunctionTok{distinct}\NormalTok{(queryItem, cog\_id, pathway\_name, root, clade\_name)}

\NormalTok{cumulative\_genes }\OtherTok{\textless{}{-}} \FunctionTok{calculate\_cumulative\_genes}\NormalTok{(nodelist) }
\NormalTok{cumulative\_bp }\OtherTok{\textless{}{-}} \FunctionTok{calculate\_cumulative\_bp}\NormalTok{(nodelist)}
  
\NormalTok{cumulative\_data }\OtherTok{\textless{}{-}} \FunctionTok{left\_join}\NormalTok{(cumulative\_genes, cumulative\_bp)}
 

\NormalTok{long\_data }\OtherTok{\textless{}{-}}\NormalTok{ cumulative\_data }\SpecialCharTok{\%\textgreater{}\%}
  \FunctionTok{pivot\_longer}\NormalTok{(}\AttributeTok{cols =} \DecValTok{5}\SpecialCharTok{:}\DecValTok{7}\NormalTok{, }
               \AttributeTok{names\_to =} \StringTok{"Process"}\NormalTok{, }
               \AttributeTok{values\_to =} \StringTok{"Value"}\NormalTok{)}

\CommentTok{\#a \textless{}{-}}
\FunctionTok{ggplot}\NormalTok{() }\SpecialCharTok{+}
  \CommentTok{\# Bar chart for cumulative\_sum}
  \FunctionTok{geom\_bar}\NormalTok{(}\AttributeTok{data =}\NormalTok{ cumulative\_data, }
           \FunctionTok{aes}\NormalTok{(}\AttributeTok{x =} \FunctionTok{factor}\NormalTok{(clade\_name, }\AttributeTok{levels =}\NormalTok{ clade\_name), }\AttributeTok{y =}\NormalTok{ cumulative\_sum), }
           \AttributeTok{stat =} \StringTok{"identity"}\NormalTok{, }\AttributeTok{fill =} \StringTok{"darkgray"}\NormalTok{, }\AttributeTok{colour =} \ConstantTok{NA}\NormalTok{) }\SpecialCharTok{+}
  \FunctionTok{geom\_text}\NormalTok{(}\AttributeTok{data =}\NormalTok{ cumulative\_data, }
            \FunctionTok{aes}\NormalTok{(}\AttributeTok{x =} \FunctionTok{factor}\NormalTok{(clade\_name, }\AttributeTok{levels =}\NormalTok{ clade\_name), }\AttributeTok{y =}\NormalTok{ cumulative\_sum, }\AttributeTok{label =}\NormalTok{ cumulative\_sum), }
            \AttributeTok{vjust =} \SpecialCharTok{{-}}\FloatTok{0.5}\NormalTok{, }\AttributeTok{size =} \DecValTok{3}\NormalTok{, }\AttributeTok{color =} \StringTok{"darkgray"}\NormalTok{) }\SpecialCharTok{+}
  \FunctionTok{scale\_color\_manual}\NormalTok{(}\AttributeTok{values =}\NormalTok{ annotation\_colors) }\SpecialCharTok{+}
  
  \FunctionTok{labs}\NormalTok{(}\AttributeTok{x =} \StringTok{"Clade Name"}\NormalTok{, }\AttributeTok{y =} \StringTok{"Cumulative Sum"}\NormalTok{, }
       \AttributeTok{title =} \StringTok{"Cumulative Sum and Biological Processes"}\NormalTok{,}
       \AttributeTok{fill =} \StringTok{"Cumulative Sum"}\NormalTok{,}
       \AttributeTok{color =} \StringTok{"Biological Processes"}\NormalTok{) }\SpecialCharTok{+}
  
\NormalTok{  theme\_main }\SpecialCharTok{+}
  \FunctionTok{theme}\NormalTok{(}\AttributeTok{axis.text.x =} \FunctionTok{element\_text}\NormalTok{(}\AttributeTok{angle =} \DecValTok{90}\NormalTok{, }\AttributeTok{hjust =} \DecValTok{1}\NormalTok{))}
\end{Highlighting}
\end{Shaded}

\includegraphics{analysis/03_plotting_abundances_files/figure-pdf/unnamed-chunk-4-1.pdf}

\begin{Shaded}
\begin{Highlighting}[]
\CommentTok{\#b \textless{}{-} }
\FunctionTok{ggplot}\NormalTok{() }\SpecialCharTok{+}
  \CommentTok{\# Bar chart for cumulative\_sum}
  \FunctionTok{geom\_bar}\NormalTok{(}\AttributeTok{data =}\NormalTok{ cumulative\_data, }
           \FunctionTok{aes}\NormalTok{(}\AttributeTok{x =} \FunctionTok{factor}\NormalTok{(}\SpecialCharTok{{-}}\NormalTok{root), }\AttributeTok{y =}\NormalTok{ cumulative\_sum), }
           \AttributeTok{stat =} \StringTok{"identity"}\NormalTok{, }\AttributeTok{fill =} \StringTok{"darkgray"}\NormalTok{, }\AttributeTok{colour =} \ConstantTok{NA}\NormalTok{) }\SpecialCharTok{+}
  \FunctionTok{geom\_text}\NormalTok{(}\AttributeTok{data =}\NormalTok{ cumulative\_data, }
            \FunctionTok{aes}\NormalTok{(}\AttributeTok{x =} \FunctionTok{factor}\NormalTok{(}\SpecialCharTok{{-}}\NormalTok{root), }\AttributeTok{y =}\NormalTok{ cumulative\_sum, }\AttributeTok{label =}\NormalTok{ cumulative\_sum), }
            \AttributeTok{vjust =} \SpecialCharTok{{-}}\FloatTok{0.5}\NormalTok{, }\AttributeTok{size =} \DecValTok{3}\NormalTok{, }\AttributeTok{color =} \StringTok{"darkgray"}\NormalTok{) }\SpecialCharTok{+}
  
  \CommentTok{\# Line chart for biological processes}
  \FunctionTok{geom\_line}\NormalTok{(}\AttributeTok{data =}\NormalTok{ long\_data, }
            \FunctionTok{aes}\NormalTok{(}\AttributeTok{x =} \FunctionTok{factor}\NormalTok{(}\SpecialCharTok{{-}}\NormalTok{root), }\AttributeTok{y =}\NormalTok{ Value, }\AttributeTok{color =}\NormalTok{ Process, }\AttributeTok{group =}\NormalTok{ Process), }
            \AttributeTok{size =} \DecValTok{1}\NormalTok{) }\SpecialCharTok{+}
  
  \CommentTok{\# Use the defined color palette}
  \FunctionTok{scale\_color\_manual}\NormalTok{(}\AttributeTok{values =}\NormalTok{ annotation\_colors) }\SpecialCharTok{+}
  
  \FunctionTok{labs}\NormalTok{(}\AttributeTok{x =} \StringTok{"Clade Name"}\NormalTok{, }\AttributeTok{y =} \StringTok{"Cumulative Sum"}\NormalTok{, }
       \AttributeTok{title =} \StringTok{"Cumulative Sum and Biological Processes"}\NormalTok{,}
       \AttributeTok{fill =} \StringTok{"Cumulative Sum"}\NormalTok{,}
       \AttributeTok{color =} \StringTok{"Biological Processes"}\NormalTok{) }\SpecialCharTok{+}
  
\NormalTok{  theme\_main }\SpecialCharTok{+}
  \FunctionTok{theme}\NormalTok{(}\AttributeTok{axis.text.x =} \FunctionTok{element\_text}\NormalTok{(}\AttributeTok{angle =} \DecValTok{45}\NormalTok{, }\AttributeTok{hjust =} \DecValTok{1}\NormalTok{))}
\end{Highlighting}
\end{Shaded}

\includegraphics{analysis/03_plotting_abundances_files/figure-pdf/unnamed-chunk-4-2.pdf}

\begin{Shaded}
\begin{Highlighting}[]
\CommentTok{\#ggsave(file = "", plot=a, width=15, height=8)}
\CommentTok{\#ggsave(file = "", plot=b, width=15, height=8)}
\end{Highlighting}
\end{Shaded}

\section{Calculation of COG Abundance by Function in
Species}\label{calculation-of-cog-abundance-by-function-in-species}

This analysis is based on the premise that similar COGs share the same
biological function. Thus, it is possible to calculate the average
abundance of functions associated with COGs in each species, using the
number of COGs present in the species identified by the \emph{TaxID}.

\subsection{Calculation Steps}\label{calculation-steps}

\begin{enumerate}
\def\labelenumi{\arabic{enumi}.}
\tightlist
\item
  \textbf{Species Mapping} Each species was mapped to its respective
  clade (hierarchical level) based on the \emph{TaxID} information.

  \begin{itemize}
  \tightlist
  \item
    The \texttt{ogr@spbranches} table was used to associate species
    (\emph{ssp\_id}) with their \emph{branches} (\emph{lca}).
  \item
    The data was organized to include the order of the \emph{TaxIDs}.
  \end{itemize}
\item
  \textbf{COG Annotation}

  \begin{itemize}
  \tightlist
  \item
    The relationship between proteins and genes was enriched with
    functional information, such as COG identifiers and metabolic
    pathway names (\emph{pathway\_name}).
  \item
    Duplicates were removed to ensure that only unique information was
    considered.
  \end{itemize}
\item
  \textbf{Integration of Species Information}

  \begin{itemize}
  \tightlist
  \item
    A mapping was created between species (\emph{TaxID}) and their
    respective clades to add the relevant taxonomic information.
  \item
    The species were ordered based on their clades, to facilitate
    hierarchical analysis.
  \end{itemize}
\item
  \textbf{Calculation of Average Abundance by Function}

  \begin{itemize}
  \tightlist
  \item
    For each species and metabolic function (\emph{pathway\_name}), the
    average abundance of COGs was calculated.
  \item
    Additional information, such as species names (\emph{ncbi\_name})
    and clades (\emph{clade\_name}), was integrated into the result.
  \end{itemize}
\item
  \textbf{Abundance Adjustment (Capping)}

  \begin{itemize}
  \tightlist
  \item
    To avoid extreme values, the average abundance was adjusted:

    \begin{itemize}
    \tightlist
    \item
      Values above a limit (mean + 3 standard deviations) were truncated
      to 100.
    \item
      The adjusted values were calculated separately by metabolic
      function.
    \end{itemize}
  \end{itemize}
\end{enumerate}

\begin{Shaded}
\begin{Highlighting}[]
\NormalTok{lca\_spp }\OtherTok{\textless{}{-}}\NormalTok{ ogr}\SpecialCharTok{@}\NormalTok{spbranches }\SpecialCharTok{\%\textgreater{}\%}
  \FunctionTok{rename}\NormalTok{(}\StringTok{"taxid"} \OtherTok{=}\NormalTok{ ssp\_id, }\StringTok{"species"} \OtherTok{=}\NormalTok{ ssp\_name, }\StringTok{"lca"} \OtherTok{=} \StringTok{"branch"}\NormalTok{) }\SpecialCharTok{\%\textgreater{}\%}
  \FunctionTok{mutate}\NormalTok{(}\AttributeTok{taxid\_order =} \FunctionTok{row\_number}\NormalTok{()) }\SpecialCharTok{\%\textgreater{}\%}
\NormalTok{  dplyr}\SpecialCharTok{::}\FunctionTok{select}\NormalTok{(lca, taxid, taxid\_order)}

\NormalTok{clade\_taxids }\OtherTok{\textless{}{-}}\NormalTok{ lca\_spp}
\NormalTok{clade\_names }\OtherTok{\textless{}{-}}\NormalTok{ vroom}\SpecialCharTok{::}\FunctionTok{vroom}\NormalTok{(}\StringTok{"https://raw.githubusercontent.com/dalmolingroup/neurotransmissionevolution/ctenophora\_before\_porifera/analysis/geneplast\_clade\_names.tsv"}\NormalTok{)}

\NormalTok{cog\_annotation }\OtherTok{\textless{}{-}}\NormalTok{ map\_ids }\SpecialCharTok{\%\textgreater{}\%}
  \FunctionTok{left\_join}\NormalTok{(groot\_df, }\AttributeTok{by =} \FunctionTok{c}\NormalTok{(}\StringTok{"stringId"} \OtherTok{=} \StringTok{"protein\_id"}\NormalTok{)) }\SpecialCharTok{\%\textgreater{}\%}
  \FunctionTok{left\_join}\NormalTok{(sensorial\_genes, }\AttributeTok{by =} \FunctionTok{c}\NormalTok{(}\StringTok{"queryItem"} \OtherTok{=} \StringTok{"gene\_symbol"}\NormalTok{)) }\SpecialCharTok{\%\textgreater{}\%}
  \FunctionTok{distinct}\NormalTok{(queryItem, cog\_id, pathway\_name) }\SpecialCharTok{\%\textgreater{}\%}
\NormalTok{  dplyr}\SpecialCharTok{::}\FunctionTok{select}\NormalTok{(cog\_id, pathway\_name) }\SpecialCharTok{\%\textgreater{}\%}
  \FunctionTok{unique}\NormalTok{() }\SpecialCharTok{\%\textgreater{}\%}
  \FunctionTok{na.omit}\NormalTok{()}

\NormalTok{cog\_abundance\_by\_taxid }\OtherTok{\textless{}{-}}\NormalTok{ cogdata }\SpecialCharTok{\%\textgreater{}\%}
  \FunctionTok{filter}\NormalTok{(cog\_id }\SpecialCharTok{\%in\%}\NormalTok{ nodelist[[}\StringTok{"cog\_id"}\NormalTok{]]) }\SpecialCharTok{\%\textgreater{}\%}
  \FunctionTok{count}\NormalTok{(ssp\_id, cog\_id, }\AttributeTok{name =} \StringTok{"abundance"}\NormalTok{) }\SpecialCharTok{\%\textgreater{}\%}
  \FunctionTok{left\_join}\NormalTok{(cog\_annotation, }\AttributeTok{by =} \StringTok{"cog\_id"}\NormalTok{)}

\CommentTok{\# Mapping species to clade info}
\NormalTok{ordered\_species }\OtherTok{\textless{}{-}}\NormalTok{ string\_eukaryotes }\SpecialCharTok{\%\textgreater{}\%}
\NormalTok{  dplyr}\SpecialCharTok{::}\FunctionTok{select}\NormalTok{(taxid, ncbi\_name) }\SpecialCharTok{\%\textgreater{}\%}
  \FunctionTok{left\_join}\NormalTok{(clade\_taxids, }\AttributeTok{by =} \StringTok{"taxid"}\NormalTok{) }\SpecialCharTok{\%\textgreater{}\%}
  \FunctionTok{left\_join}\NormalTok{(clade\_names, }\AttributeTok{by =} \FunctionTok{c}\NormalTok{(}\StringTok{"lca"} \OtherTok{=} \StringTok{"root"}\NormalTok{)) }\SpecialCharTok{\%\textgreater{}\%}
  \FunctionTok{na.omit}\NormalTok{() }\SpecialCharTok{\%\textgreater{}\%} \FunctionTok{unique}\NormalTok{() }\SpecialCharTok{\%\textgreater{}\%}
  \FunctionTok{arrange}\NormalTok{(}\FunctionTok{desc}\NormalTok{(lca)) }\SpecialCharTok{\%\textgreater{}\%}
\NormalTok{  dplyr}\SpecialCharTok{::}\FunctionTok{select}\NormalTok{(}\SpecialCharTok{{-}}\NormalTok{taxid\_order)}
  
\NormalTok{avg\_abundance\_by\_function }\OtherTok{\textless{}{-}}\NormalTok{ cog\_abundance\_by\_taxid }\SpecialCharTok{\%\textgreater{}\%}
  \FunctionTok{group\_by}\NormalTok{(ssp\_id, pathway\_name) }\SpecialCharTok{\%\textgreater{}\%}
  \FunctionTok{summarise}\NormalTok{(}\AttributeTok{avg\_abundance =} \FunctionTok{mean}\NormalTok{(abundance)) }\SpecialCharTok{\%\textgreater{}\%}
  \FunctionTok{ungroup}\NormalTok{() }\SpecialCharTok{\%\textgreater{}\%}
  \CommentTok{\# Adding species and clade info}
  \FunctionTok{left\_join}\NormalTok{(ordered\_species }\SpecialCharTok{\%\textgreater{}\%} \FunctionTok{mutate}\NormalTok{(}\AttributeTok{taxid =} \FunctionTok{as.double}\NormalTok{(taxid)), }\AttributeTok{by =} \FunctionTok{c}\NormalTok{(}\StringTok{"ssp\_id"} \OtherTok{=} \StringTok{"taxid"}\NormalTok{)) }\SpecialCharTok{\%\textgreater{}\%}
  \FunctionTok{unique}\NormalTok{() }\SpecialCharTok{\%\textgreater{}\%}
  \FunctionTok{arrange}\NormalTok{(}\FunctionTok{desc}\NormalTok{(lca)) }\SpecialCharTok{\%\textgreater{}\%}
  \FunctionTok{mutate}\NormalTok{(}\AttributeTok{ncbi\_name =} \FunctionTok{factor}\NormalTok{(ncbi\_name, }\AttributeTok{levels =} \FunctionTok{unique}\NormalTok{(ncbi\_name)),}
         \AttributeTok{clade\_name =} \FunctionTok{factor}\NormalTok{(clade\_name, }\AttributeTok{levels =} \FunctionTok{unique}\NormalTok{(clade\_name))) }\SpecialCharTok{\%\textgreater{}\%}
  \FunctionTok{na.omit}\NormalTok{()}

\NormalTok{capped\_abundance\_by\_function }\OtherTok{\textless{}{-}}\NormalTok{ avg\_abundance\_by\_function }\SpecialCharTok{\%\textgreater{}\%}
  \CommentTok{\# mutate(capped\_abundance = ifelse(abundance \textgreater{}= 100, 100, abundance)) \%\textgreater{}\%}
  \FunctionTok{group\_by}\NormalTok{(pathway\_name) }\SpecialCharTok{\%\textgreater{}\%}
  \FunctionTok{mutate}\NormalTok{(}
    \CommentTok{\# max\_abundance = max(abundance[lca \textless{}= 29])}
    \AttributeTok{max\_abundance =}\NormalTok{ avg\_abundance[lca }\SpecialCharTok{\textless{}=} \DecValTok{29}\NormalTok{] }\SpecialCharTok{\%\textgreater{}\%}\NormalTok{ \{ }\FunctionTok{mean}\NormalTok{(.) }\SpecialCharTok{+} \DecValTok{3}\SpecialCharTok{*}\FunctionTok{sd}\NormalTok{(.) \}}
\NormalTok{    ,}\AttributeTok{abundance     =} \FunctionTok{ifelse}\NormalTok{(avg\_abundance }\SpecialCharTok{\textgreater{}=}\NormalTok{ max\_abundance, }\FunctionTok{pmin}\NormalTok{(max\_abundance, }\DecValTok{100}\NormalTok{), }\FunctionTok{pmin}\NormalTok{(avg\_abundance, }\DecValTok{100}\NormalTok{)))}

\CommentTok{\# List of signatures}
\NormalTok{signatures }\OtherTok{\textless{}{-}} \FunctionTok{unique}\NormalTok{(node\_annotation}\SpecialCharTok{$}\NormalTok{pathway\_name)}

\NormalTok{roots\_seq }\OtherTok{\textless{}{-}}\NormalTok{ node\_annotation }\SpecialCharTok{\%\textgreater{}\%}
  \FunctionTok{arrange}\NormalTok{(}\FunctionTok{desc}\NormalTok{(root)) }\SpecialCharTok{\%\textgreater{}\%}
\NormalTok{  dplyr}\SpecialCharTok{::} \FunctionTok{select}\NormalTok{(root, clade\_name) }\SpecialCharTok{\%\textgreater{}\%}
  \FunctionTok{unique}\NormalTok{()}

\NormalTok{roots\_seq}\SpecialCharTok{$}\NormalTok{clade\_name }\OtherTok{\textless{}{-}} \FunctionTok{factor}\NormalTok{(roots\_seq}\SpecialCharTok{$}\NormalTok{clade\_name, }\AttributeTok{levels =}\NormalTok{ roots\_seq}\SpecialCharTok{$}\NormalTok{clade\_name)}
\end{Highlighting}
\end{Shaded}

\subsection{Visualization}\label{visualization}

In the following graphs, it is possible to observe the average abundance
of COGs in each clade. Each bar represents a species within the
respective clade, while the colors indicate the different metabolic
functions associated with the orthologous groups.

\subsubsection{Graph of Average COG Abundance by
Clade}\label{graph-of-average-cog-abundance-by-clade}

\begin{itemize}
\tightlist
\item
  The \textbf{X} axis represents the species.
\item
  The \textbf{Y} axis indicates the average abundance of proteins in
  orthologous groups.
\item
  The bars are colored according to the metabolic function
  (\emph{pathway\_name}).
\end{itemize}

\subsubsection{Graph with Adjusted Abundance
(Capped)}\label{graph-with-adjusted-abundance-capped}

\begin{itemize}
\tightlist
\item
  To avoid distortions caused by extreme values, the abundance was
  adjusted to a maximum limit (mean + 3 standard deviations).
\item
  This graph provides a more balanced visualization, especially for
  metabolic functions with very high values.
\end{itemize}

\begin{Shaded}
\begin{Highlighting}[]
\CommentTok{\# Plotting by species}
\FunctionTok{ggplot}\NormalTok{(avg\_abundance\_by\_function) }\SpecialCharTok{+}
  \CommentTok{\# Geoms  {-}{-}{-}{-}{-}{-}{-}{-}{-}{-}{-}{-}{-}{-}{-}{-}}
\NormalTok{choanoflagellata\_line }\SpecialCharTok{+}
  \FunctionTok{geom\_bar}\NormalTok{(}
    \FunctionTok{aes}\NormalTok{(}\AttributeTok{x =}\NormalTok{ ncbi\_name, }\AttributeTok{y =}\NormalTok{ avg\_abundance, }\AttributeTok{fill =}\NormalTok{ pathway\_name, }\AttributeTok{color =} \FunctionTok{after\_scale}\NormalTok{(}\FunctionTok{darken}\NormalTok{(fill, }\FloatTok{0.1}\NormalTok{)))}
\NormalTok{    ,}\AttributeTok{stat =} \StringTok{"identity"}
\NormalTok{  ) }\SpecialCharTok{+}
  \CommentTok{\# Labels  {-}{-}{-}{-}{-}{-}{-}{-}{-}{-}{-}{-}{-}{-}{-}}
 \FunctionTok{xlab}\NormalTok{(}\StringTok{"Species"}\NormalTok{) }\SpecialCharTok{+}
  \FunctionTok{ylab}\NormalTok{(}\StringTok{"Average protein abundance in orthologous groups"}\NormalTok{) }\SpecialCharTok{+}
  \CommentTok{\#ylab("Average protein abundance in orthologous groups") +}
  \CommentTok{\# Scales {-}{-}{-}{-}{-}{-}{-}{-}{-}{-}{-}{-}{-}{-}{-}{-}}
\FunctionTok{scale\_y\_continuous}\NormalTok{(}\AttributeTok{breaks =}\NormalTok{ tick\_function, }\AttributeTok{minor\_breaks =} \ConstantTok{NULL}\NormalTok{) }\SpecialCharTok{+}
  \FunctionTok{scale\_fill\_manual}\NormalTok{(}\AttributeTok{values =}\NormalTok{ annotation\_colors }\SpecialCharTok{\%\textgreater{}\%} \FunctionTok{darken}\NormalTok{(}\FloatTok{0.1}\NormalTok{)) }\SpecialCharTok{+}
  \CommentTok{\# Styling {-}{-}{-}{-}{-}{-}{-}{-}{-}{-}{-}{-}{-}{-}{-}}
\FunctionTok{facet\_grid}\NormalTok{(}
\NormalTok{  pathway\_name }\SpecialCharTok{\textasciitilde{}}\NormalTok{ clade\_name}
\NormalTok{  ,}\AttributeTok{scales   =} \StringTok{"free"}
\NormalTok{  ,}\AttributeTok{space    =} \StringTok{"free"}
\NormalTok{  ,}\AttributeTok{labeller =} \FunctionTok{labeller}\NormalTok{(}\AttributeTok{annotation =}\NormalTok{ annotation\_labels)}
\NormalTok{) }\SpecialCharTok{+}
  \FunctionTok{theme\_classic}\NormalTok{() }\SpecialCharTok{+} 
\NormalTok{  theme\_main}
\end{Highlighting}
\end{Shaded}

\includegraphics{analysis/03_plotting_abundances_files/figure-pdf/unnamed-chunk-6-1.pdf}

\begin{Shaded}
\begin{Highlighting}[]
\CommentTok{\# Plotting by species capped}
\FunctionTok{ggplot}\NormalTok{(capped\_abundance\_by\_function) }\SpecialCharTok{+}
  \CommentTok{\# Geoms  {-}{-}{-}{-}{-}{-}{-}{-}{-}{-}{-}{-}{-}{-}{-}{-}}
\NormalTok{choanoflagellata\_line }\SpecialCharTok{+}
  \FunctionTok{geom\_bar}\NormalTok{(}
    \FunctionTok{aes}\NormalTok{(}\AttributeTok{x =}\NormalTok{ ncbi\_name, }\AttributeTok{y =}\NormalTok{ abundance, }\AttributeTok{fill =}\NormalTok{ pathway\_name, }\AttributeTok{color =} \FunctionTok{after\_scale}\NormalTok{(}\FunctionTok{darken}\NormalTok{(fill, }\FloatTok{0.1}\NormalTok{)))}
\NormalTok{    ,}\AttributeTok{stat =} \StringTok{"identity"}
\NormalTok{  ) }\SpecialCharTok{+}
  \CommentTok{\# Labels  {-}{-}{-}{-}{-}{-}{-}{-}{-}{-}{-}{-}{-}{-}{-}}
  \FunctionTok{xlab}\NormalTok{(}\StringTok{"Species"}\NormalTok{) }\SpecialCharTok{+}
  \FunctionTok{ylab}\NormalTok{(}\StringTok{"Average protein abundance in orthologous groups"}\NormalTok{) }\SpecialCharTok{+}
  \CommentTok{\#ylab("Average protein abundance in orthologous groups") +}
  \CommentTok{\# Scales {-}{-}{-}{-}{-}{-}{-}{-}{-}{-}{-}{-}{-}{-}{-}{-}}
  \FunctionTok{scale\_y\_continuous}\NormalTok{(}\AttributeTok{breaks =}\NormalTok{ tick\_function, }\AttributeTok{minor\_breaks =} \ConstantTok{NULL}\NormalTok{) }\SpecialCharTok{+}
  \FunctionTok{scale\_fill\_manual}\NormalTok{(}\AttributeTok{values =}\NormalTok{ annotation\_colors }\SpecialCharTok{\%\textgreater{}\%} \FunctionTok{darken}\NormalTok{(}\FloatTok{0.1}\NormalTok{)) }\SpecialCharTok{+}
  \CommentTok{\# Styling {-}{-}{-}{-}{-}{-}{-}{-}{-}{-}{-}{-}{-}{-}{-}}
\FunctionTok{facet\_grid}\NormalTok{(}
\NormalTok{  pathway\_name }\SpecialCharTok{\textasciitilde{}}\NormalTok{ clade\_name}
\NormalTok{  ,}\AttributeTok{scales   =} \StringTok{"free"}
\NormalTok{  ,}\AttributeTok{space    =} \StringTok{"free"}
\NormalTok{  ,}\AttributeTok{labeller =} \FunctionTok{labeller}\NormalTok{(}\AttributeTok{annotation =}\NormalTok{ annotation\_labels)}
\NormalTok{) }\SpecialCharTok{+}
  \FunctionTok{theme\_classic}\NormalTok{() }\SpecialCharTok{+} 
\NormalTok{  theme\_main}
\end{Highlighting}
\end{Shaded}

\includegraphics{analysis/03_plotting_abundances_files/figure-pdf/unnamed-chunk-6-2.pdf}

\subsubsection{Simplified Visualization: Abundance by
Clade}\label{simplified-visualization-abundance-by-clade}

For a less detailed analysis, it is possible to visualize the average
abundance of COGs only at the clade level, ignoring the differences
between individual species. This graph provides a more direct overview,
ideal for identifying global trends.

\begin{itemize}
\tightlist
\item
  The \textbf{X} axis represents the clades.
\item
  The \textbf{Y} axis shows the average abundance of proteins in
  orthologous groups per clade.
\item
  The bars are grouped by clade and colored according to the metabolic
  functions (\emph{pathway\_name}).
\end{itemize}

\begin{Shaded}
\begin{Highlighting}[]
\CommentTok{\# Ploting by clade}
\FunctionTok{ggplot}\NormalTok{(avg\_abundance\_by\_function) }\SpecialCharTok{+}
  \FunctionTok{geom\_bar}\NormalTok{(}
    \FunctionTok{aes}\NormalTok{(}\AttributeTok{x =}\NormalTok{ clade\_name, }\AttributeTok{y =}\NormalTok{ avg\_abundance, }\AttributeTok{fill =}\NormalTok{ pathway\_name, }\AttributeTok{color =} \FunctionTok{after\_scale}\NormalTok{(}\FunctionTok{darken}\NormalTok{(fill, }\FloatTok{0.1}\NormalTok{)))}
\NormalTok{    ,}\AttributeTok{stat =} \StringTok{"summary"}
\NormalTok{    ,}\AttributeTok{fun  =} \StringTok{"mean"}
\NormalTok{  ) }\SpecialCharTok{+}
  \FunctionTok{scale\_y\_continuous}\NormalTok{(}\AttributeTok{breaks =}\NormalTok{ tick\_function, }\AttributeTok{minor\_breaks =} \ConstantTok{NULL}\NormalTok{) }\SpecialCharTok{+}
  \FunctionTok{scale\_fill\_manual}\NormalTok{(}\AttributeTok{values =}\NormalTok{ annotation\_colors, }\AttributeTok{guide =} \StringTok{"none"}\NormalTok{) }\SpecialCharTok{+}
  \FunctionTok{facet\_grid}\NormalTok{(}
\NormalTok{    pathway\_name }\SpecialCharTok{\textasciitilde{}}\NormalTok{ .}
\NormalTok{    ,}\AttributeTok{scales   =} \StringTok{"free"}
\NormalTok{    ,}\AttributeTok{space    =} \StringTok{"free\_y"}
\NormalTok{    ,}\AttributeTok{labeller =} \FunctionTok{labeller}\NormalTok{(}\AttributeTok{annotation =} \FunctionTok{sub}\NormalTok{(}\StringTok{"}\SpecialCharTok{\textbackslash{}n}\StringTok{"}\NormalTok{, }\StringTok{""}\NormalTok{, annotation\_labels))}
\NormalTok{  ) }\SpecialCharTok{+}
  \FunctionTok{xlab}\NormalTok{(}\StringTok{"Clades"}\NormalTok{) }\SpecialCharTok{+}
  \FunctionTok{ylab}\NormalTok{(}\StringTok{"Average abundance by clade"}\NormalTok{) }\SpecialCharTok{+}
  \FunctionTok{theme\_classic}\NormalTok{() }\SpecialCharTok{+} 
\NormalTok{  theme\_average}
\end{Highlighting}
\end{Shaded}

\includegraphics{analysis/03_plotting_abundances_files/figure-pdf/unnamed-chunk-7-1.pdf}

\bookmarksetup{startatroot}

\chapter{OrthoGuide: Pre-computed Gene Rooting
Data}\label{orthoguide-pre-computed-gene-rooting-data}

\bookmarksetup{startatroot}

\chapter{OrthoGuide: Pre-computed Gene Rooting
Data}\label{orthoguide-pre-computed-gene-rooting-data-1}

OrthoGuide is available at
\url{https://dalmolingroup.imd.ufrn.br/orthoguide/} and provides
immediate access to high-quality, pre-computed rooting data for 8
species. Instead of running complex analyses, users can query this
database and get the information they need in seconds, along with a few
exploratory plots.

\section{How OrthoGuide Works}\label{how-orthoguide-works}

All data in the database is generated using the
\href{https://github.com/dalmolingroup/genebridge}{GeneBridge R package
(v0.99.2)}. The gene-to-COG (Clusters of Orthologous Groups) information
was obtained from STRING-db version 11.0.



\end{document}
